\chapter{Fazit}
\label{chapter:6}

Wir haben in dieser Arbeit mit der Motivation für Regression begonnen. Danach haben wir die mathematischen Grundlagen der Regressionsanalyse erläutert. Darauf aufbauend haben wir die Umsetzung in R, TensorFlow und SQL demonstriert.

Hierbei haben zwei verschiedene Arten zur Berechnung der Parameter verwendet. Bei expliziten Berechnungen wurden verschiedene Berechnungsformeln ausgewertet. Bei iterativer Berechnung wurden Optimierungsverfahren verwendet, um die Werte der gesuchten Parameter Schritt für Schritt besser zu approximieren.

% TODO: glm Implementierung beschreiben
In R haben wir ausschließlich explizite Berechnungen durchgeführt. Dazu haben wir die vorhandenen Funktionen $lm$ und $glm$ verwendet. Die Implementierungen der beiden Funktionen wurden an die gleichnamigen Funktionen in der verwandten Programmiersprache S angelehnt. Mehr dazu findet man in \cite{statistical}.

In TensorFlow kamen dagegen ausschließlich iterative Berechnungen zum Einsatz. Wir haben ein von TensorFlow implementiertes Gradientenverfahren genutzt, um eine von uns definierte Kostenfunktion zu minimieren. Die Kostenfunktionen waren dabei die Summe der kleinsten Quadrate bei linearer Regression und die inverse Likelihoodfunktion bei logistischer Regression.

In SQL haben wir beide Berechnungsarten umgesetzt. Bei linearer Regression haben wir die expliziten Formeln aus den Kapiteln \ref{subsection:2:1:1} und \ref{subsection:2:1:2} verwendet. Bei logistischer Regression wurde wiederum ein Gradientenverfahren angewandt. Dieses Mal wurde das Verfahren eigens implementiert.

Danach haben wir die Implementierungen bezüglich der Laufzeit miteinander verglichen. Dabei haben wir festgestellt, dass die SQL-Skripte bei kleinen Datenmengen und bei Verwendung von expliziten Berechnungsformeln durchaus mit R mithalten können. Besonders bei logistischer Regression erkannte man aber, dass die Implementierung in R von der Laufzeit her deutlich überlegen ist. TensorFlow war durch die strikte Verwendung iterativer Berechnungsarten immer langsamer als R.

Eine Implementierung der hier gezeigten Funktionen für Regression direkt in Datenbanksystemen und die Verwendung von effizienteren Algorithmen könnte die Performance noch steigern und würde Regressionsanalyse in SQL auch praktisch nutzbar machen. Diese Arbeit war ein erster "Proof of Concept" zu diesem Thema.
