\chapter{Grundlagen statistischer Methoden}
\label{chapter:2}

Bei der Regressionsanalyse geht es im Allgemeinen darum, das Verhalten einer Größe $Y$ in Abhängigkeit einer oder mehrerer anderer Größen $X_1, X_2, \dots, X_n$ zu modellieren. Die Größe $Y$ wird abhängig genannt, die Größen $X_i$ nennt man unabhängig. Für diese Arbeit wollen wir zunächst einige Annahmen treffen. Diese sollen immer gelten, falls nicht explizit etwas anderes festgelegt wird.
\begin{itemize}
    \item Die genannten Größen sind Zufallsvariablen. Eine solche Zufallsvariable ist eine Funktion deren Werte die Ergebnisse eines Zufallsvorgangs darstellt.
    \item Die Zufallsvariablen sind auf der Menge $M = \{1, \dots, m\}$ definiert und bilden in die reellen Zahlen ab:
    \begin{align*}
        Y: M \rightarrow \mathbb{R},~~ X_1: M \rightarrow \mathbb{R},~~ \dots,~~ X_n: M \rightarrow \mathbb{R}
    \end{align*}
    Das bedeutet die Zufallsvariablen sind metrisch skaliert. Die $m$ Zahlen in der Menge $M$ entsprechen den $m$ Datenpunkten, die wir als Datenbasis für die Regressionsanalyse besitzen.
    \item Wir verwenden die folgenden Abkürzungen für die Werte der Zufallsvariablen:
    \begin{align*}
        y_i &:= Y(i) ~~~\text{für alle}~ i \in M,\\
        x_{i, j} &:= X_j(i) ~~~\text{für alle}~ i \in M ~~\text{und}~ 1 \leq j \leq n
    \end{align*}
    \item Einen Datenpunkt aus unserer Datenbasis fassen wir als Vektor der Länge $(n + 1)$ auf. Damit lässt sich die Datenbasis schreiben als:
    \begin{align*}
        (y_1, x_{1,1}, \dots, x_{1,n}), \dots, (y_m, x_{m,1}, \dots, x_{m,n})
    \end{align*}
\end{itemize}
Nun definieren wir ein Modell, mit dem der Zusammenhang zwischen der abhängigen und den unabhängigen Variablen dargestellt werden soll. Dazu verwenden wir eine Funktion $f$, welche für Werte von  $X_1$ bis $X_n$ einen geschätzten Wert für $Y$ liefert. Idealerweise existiert eine Funktion, die zum einen eine einfache Darstellung (z.B. durch eine arithmetische Formel) besitzt und zum anderen alle unabhängigen Werte der Datenmenge exakt prognostiziert. Das bedeutet:
\begin{align*}
    y_i = f(x_{i, 1}, \dots, x_{i, n}) ~~~\text{für alle}~~ 1 \leq i \leq m
\end{align*}
Im Allgemeinen ist es nicht möglich eine solche Funktion zu finden. Man verwirft also die Anforderung der Exaktheit für alle Datenpunkte und versucht stattdessen eine einfache Funktion zu finden, die die Datenmenge möglichst gut approximiert. Wir definieren für jeden Datenpunkt den Fehler $e_i$, der sich durch die Modellfunktion $f$ ergibt:
\begin{align*}
    e_i := y_i - f(x_{i, 1}, \dots, x_{i, n})
\end{align*}
Je näher ein Fehlerterm bei null liegt, desto besser ist die Annäherung. Um eine gute Approximation für die gesamte Datenmenge zu erhalten, sollten also alle diese Fehlerterme möglichst klein bleiben. Das Ziel der Regressionsanalyse ist nun die Bestimmung der Funktion $f$. Dazu nimmt man in der Regel an, dass $f$ eine bestimmte Form hat. Diese Annahme kann sich je nach der Problemstellung unterscheiden. Oft arbeitet man mit linearen Funktionen $f$, aber auch quadratische, exponentielle oder logistische Funktionen sind nicht unüblich.

\section{Lineare Regression}
\label{section:2:1}

Bei der linearen Regression geht man von einem linearen Zusammenhang zwischen der abhängigen und den unabhängigen Variablen aus. Die Funktion $f$ ist also von folgender Form:
\begin{align*}
    f(x_1, \dots, x_n) = \alpha + \sum_{j=1}^n \beta_j \cdot x_j ~~~ \text{mit} ~~ \beta_j \in \mathbb{R}
\end{align*}
Dabei sind $\alpha$ und $\beta_k$ für $k = 1, \dots, n$ reelle Zahlen, die sogenannten Parameter der Funktion. Das Maß für die Qualität von $f$ ist die Summe der quadrierten Fehlerterme. Diese Summe kann wiederum als Funktion in Abhängigkeit der Parameter definiert werden:
\begin{align*}
    E \left( \alpha, \beta_1, \dots, \beta_n \right) := \sum_{i=1}^m e_i^2 =  \sum_{i=1}^m \big( y_i - f(x_{i, 1}, \dots, x_{i, n}) \big)^2 = \sum_{i=1}^m \left( y_i - \alpha - \sum_{j=1}^n \beta_j \cdot x_{i, j} \right)^2
\end{align*}
Bei der linearen Regression suchen die Parameter $\hat\alpha$ und $\hat\beta_k$ für die gilt:
\begin{align*}
    E \left( \hat\alpha, \hat\beta_1, \dots, \hat\beta_n \right) = \min \big\{E(\alpha, \beta_1, \dots, \beta_n) ~\big|~ \alpha \in \mathbb{R}, \beta_1 \in \mathbb{R}, \dots, \beta_n \in \mathbb{R} \big\}
\end{align*}
Um dieses Minimierungsproblem zu lösen berechnen wir die partiellen Ableitungen von $E$ nach allen Parametern:
\begin{align*}
    \dfrac{\partial E}{\partial \alpha} &= \sum_{i=1}^m 2 \cdot \left( y_i - \alpha - \sum_{j=1}^n \beta_j \cdot x_{i, j} \right) \cdot (- 1) \\
    &= - 2 \cdot \sum_{i=1}^m \left( y_i - \alpha - \sum_{j=1}^n \beta_j \cdot x_{i, j} \right) \\
    \dfrac{\partial E}{\partial \beta_k} &= \sum_{i=1}^m 2 \cdot \left( y_i - \alpha - \sum_{j=1}^n \beta_j \cdot x_{i, j} \right) \cdot (- x_{i, k}) \\
    &= - 2 \cdot \sum_{i=1}^m x_{i, k} \cdot \left( y_i - \alpha - \sum_{j=1}^n \beta_j \cdot x_{i, j} \right) \\
\end{align*}
Durch Nullsetzen der partiellen Ableitungen erhält man ein lineares Gleichungssystem mit $(n + 1)$ Gleichungen und ebenso vielen Unbekannten.
\begin{align*}
    \dfrac{\partial E}{\partial \alpha} = 0, ~~ \dfrac{\partial E}{\partial \beta_1} = 0, ~~ \dots, ~~ \dfrac{\partial E}{\partial \beta_n} = 0
\end{align*}
Die Lösung dieses Gleichungssystems (falls diese existiert) ist das gesuchte Minimum. Damit hat man also die gesuchten Parameter für die lineare Funktion $f$ gefunden.

\subsection{Einfache lineare Regression}
\label{subsection:2:1:1}

Man spricht von einfacher linearer Regression, wenn man mit nur eine unabhängige Variable arbeitet. Anschaulich möchte mann hier die bestmögliche Schätzgerade durch eine gegebene Punktwolke legen.

Wir nennen die unabhängige Variable in diesem Kapitel statt $X_1$ einfach nur $X$. Ebenso schreiben wir $\beta := \beta_1$ und $x_i := x_{i, 1}$. Dann können wir das lineare Gleichungssystem zum Auffinden des Minimums wie folgt explizit aufschreiben:
\begin{align*}
    0 &= - 2 \cdot \sum_{i=1}^m (y_i - \alpha - \beta \cdot x_i)\\
    0 &= - 2 \cdot \sum_{i=1}^m x_i \cdot (y_i - \alpha - \beta \cdot x_i)
\end{align*}
Dieses Gleichungssystem kann explizit gelöst werden. Man erhält das folgende Ergebnis für $\hat\alpha$ und $\hat\beta$:
\begin{align*}
    \hat\beta &= \dfrac{\sum\limits_{i=1}^m (x_i - \bar x)(y_i - \bar y)}{\sum\limits_{i=1}^m (x_i - \bar x)^2}\\
    \hat\alpha &= \bar y - \hat\beta \bar x
\end{align*}
Dabei bezeichnen $\bar x$ und $\bar y$ die Mittelwerte von $X$ respektive $Y$, also:
\begin{align*}
    \bar x = \dfrac{1}{m} \sum_{i=1}^m x_i ~~,~~~ \bar y &= \dfrac{1}{m} \sum_{i=1}^m y_i
\end{align*}
Die Herleitung dieser Lösungen findet sich in [*zitiere ein Statistik-Buch*].

\subsection{Multiple lineare Regression}
\label{subsection:2:1:2}

Bei multipler linearer Regression existieren mindestens zwei unabhängige Variablen. Hier ist es nicht mehr zweckmäßig eine explizite Lösung anzugeben. Nun sind alternative Methoden zur Berechnung der Parameter nötig.

Neben einer Vielzahl von Algorithmen, die ein Optimierungsproblem iterativ lösen, gibt es auch die Möglichkeit die Parameter durch Matrizenmultiplikation zu berechnen. Definieren wir dazu die folgenden Matrizen und Vektoren:
\begin{align*}
    X &= \begin{pmatrix}
        1 & x_{1, 1} & \dots & x_{1, n} \\
        \vdots & \vdots & \ddots & \vdots \\
        1 & x_{m, 1} & \dots & x_{m, n}
    \end{pmatrix} \in \mathbb{R}^{m \times (n + 1)} \\
    y &= \begin{pmatrix}
        y_1 \\
        \vdots \\
        y_m
    \end{pmatrix} \in \mathbb{R}^{m \times 1}, ~~~
    b = \begin{pmatrix}
        \hat\alpha \\
        \hat\beta_1 \\
        \vdots \\
        \hat\beta_n
    \end{pmatrix} \in \mathbb{R}^{(n + 1) \times 1}
\end{align*}
Dabei ist $b$ der Vektor mit den gesuchten Parametern für die Minimierung der kleinsten Quadrate bzw. der Funktion $E$. Falls die Matrix $X^T X$ invertiertbar ist, gilt die folgende Formel für die Berechnung der gesuchten Parameter:
\begin{align*}
    b = (X^T X)^{-1} X^T y
\end{align*}

\section{Logistische Regression}
\label{section:2:2}

Die logistische Regression findet Anwendung im Falle, dass die abhängige Variable eine binäre Variable ist, also eine Variable, die nur zwei Werte annehmen kann. Oft handelt es sich um eine Eigenschaft oder einen Gegenstand, den man entweder besitzt oder nicht, wie zum Beispiel ein Premium-Abonnement für eine Web-Service oder der Besitz eines Auto. Auch das Geschlecht einer Person ist ein Beispiel für eine binäre Variable. Wir bezeichnen die beiden möglichen Werte einer solchen Variablen mit $0$ und $1$. Die Zuordnung vom Merkmal zur Zahl ist frei wählbar.

Lineare Regression eignet sich nicht zur Modellierung einer binären Variablen, da eine lineare Funktion entweder konstant oder unbeschränkt ist, in zweiten Fall also insbesondere Werte größer als 1 und kleiner als 0 annimmt. Um diesem Problem abzuhelfen wählen wir zusätzlich zu der linearen Funktion eine weitere Funktion, die beliebige Zahlen auf das Interval $[0, 1]$ abbildet. Im Falle der logistischen Regression verwendet man die gleichnamige logistische Funktion:
\begin{align*}
    l: \mathbb{R} \rightarrow (0, 1),~~ x \mapsto \dfrac{1}{1+e^{-x}}
\end{align*}
Diese Funktion wendet man nun auf die Linearkombination aller unabhängigen Variablen mit Parametern $\beta_1$ bis $\beta_n$ und konstantem Term $\alpha$ an. Zur Vereinfachung definieren wir für das restliche Kapitel die Variable $c_i$ wie folgt:
\begin{align*}
    c_i := \alpha + \sum_{j=1}^n \beta_j \cdot x_{i, j}
\end{align*}
Das Ergebnis der Funktion $l$ für den $i$-ten Datensatz bezeichnen wir mit $\pi_i$. Dieses ist wieder eine Funktion in Abhängigkeit der Parameter:
\begin{align*}
    \pi_i = \pi_i(\alpha, \beta_1, \dots, \beta_n) := l \left( \alpha + \sum_{j=1}^n \beta_j \cdot x_{i, j} \right) = \dfrac{1}{1 + e^{-c_i}}
\end{align*}
Wir stellen hierbei fest, dass folgende Identität für die Funktionen $\pi_i$ gilt:
\begin{align*}
    \pi_i(- \alpha, - \beta_1, \dots, - \beta_n) &= \dfrac{1}{1 + e^{c_i}} = \dfrac{1 + e^{c_i} - e^{c_i}}{1 + e^{c_i}} \\
    &= 1 - \dfrac{e^{c_i}}{1 + e^{c_i}} = 1 - \dfrac{1}{e^{-c_i} + 1} \\
    &= 1 - \pi_i(\alpha, \beta_1, \dots, \beta_n)
\end{align*}
Wir interpretieren $\pi_i$ als die Wahrscheinlichkeit dafür, dass die abhängige Variable eines Datensatzes mit unabhängigen Variablen $x_{i, 1}, \dots, x_{i, n}$ gleich $1$ ist, also:
\begin{align*}
    \pi_i = P(Y_i = 1 | X_1 = x_{i, 1}, \dots, X_n = x_{i, n})
\end{align*}
Man möchte die Parameter $\alpha$ und $\beta_k$ nun so schätzen, dass die Wahrscheinlichkeit für das Auftreten der vorhandenen Datenbasis maximiert wird. Diese Wahrscheinlichkeit ist gegeben durch:
\begin{align*}
    L(\alpha, \beta_1, \dots, \beta_n) &= \prod_{i=1}^m P(Y_i = y_i | X_1 = x_{i, 1}, \dots, X_n = x_{i, n}) \\
    &= \prod_{i=1}^m y_i \cdot \pi_i(\alpha, \beta_1, \dots, \beta_n) + (1 - y_i) \cdot (1 - \pi_i(\alpha, \beta_1, \dots, \beta_n)) \\
    &= \prod_{i=1}^m y_i \cdot \pi_i(\alpha, \beta_1, \dots, \beta_n) + (1 - y_i) \cdot \pi_i(- \alpha, - \beta_1, \dots, - \beta_n))
\end{align*}
Da alle $y_i$ im Fall der logistischen Regression entweder gleich $0$ oder gleich $1$ sind, ist immer nur einer der beiden Summanden in jedem Faktor nicht null. Diese Fallunterscheidung kann man auch in das Vorzeichen der Parameter verschieben, da sich die beiden möglichen Faktoren nur darin unterscheiden. Dann erhält man:
\begin{align*}
    L(\alpha, \beta_1, \dots, \beta_n) = \prod_{i=1}^m \pi(&(2 \cdot y_i - 1) \cdot \alpha, \\
    &(2 \cdot y_i - 1) \cdot \beta_1, \\
    &\dots, \\
    &(2 \cdot y_i - 1) \cdot \beta_n)
\end{align*}
Das Verfahren der Maximierung dieser Wahrscheinlichkeit bezeichnet man auch als Maximum-Likelihood-Methode. Die Funktion $L$ nennt man Likelihoodfunktion. Oft maximiert man nicht $L$ direkt, sondern eher $L_{log} := \ln(L)$. Der Sinn ist, dass man das Produkt damit in eine Summe einzelner Logarithmen umwandeln kann, welche wiederum einfacher abzuleiten ist. Die Maximierung von $L$ ist äquivalent mit der von $L_{log}$, da der Logarithmus eine stetig wachsende Funktion ist. Die Werte von $L$ liegen stets zwischen $0$ und $1$, also ist $L_{log}$ wohldefiniert.

Um dieses Regressionsproblem zu lösen, muss man also die Parameter $\hat\alpha, \hat\beta_1, \dots, \hat\beta_n$ finden, für die gilt:
\begin{align*}
    L(\hat\alpha, \hat\beta_1, \dots, \hat\beta_n) = \max \left\{ L(\alpha, \beta_1, \dots, \beta_n) ~|~ \alpha \in \mathbb{R}, \beta_1 \in \mathbb{R}, \dots, \beta_n \in \mathbb{R} \right\}
\end{align*}
In diesem Fall kommt man nicht mehr an einer iterativen Lösung vorbei, da das lineare Gleichungssystem aus den partiellen Ableitungen nicht mehr exakt lösbar ist. Eine der einfachsten Methoden zur Lösung von Optimierungsproblemen ist das Gradientenverfahren, welches im kommenden Teilkapitel kurz eingeführt wird. Danach wird gezeigt, wie man das Gradientenverfahren für logistische Regression anwendet.

\subsection{Gradientenverfahren}
\label{subsection:2:2:1}

Das Gradientenverfahren ist ein iterativer Algorithmus zur Lösung von Optimierungsproblemen. Nachdem wir hier bei der logistischen Regression eine Funktion maximieren wollen führen wir das Gradientenverfahren dementsprechend als Maximierungsalgorithmus ein. Man kann dasselbe Verfahren aber auch zur Lösung von Minimierungsproblemen einsetzen. Gegeben sei also eine Funktion der folgenden Form, die maximiert werden soll:
\begin{align*}
    L: \mathbb{R}^{n+1} \rightarrow \mathbb{R},~~ (\alpha, \beta_1, \dots, \beta_n) \mapsto L(\alpha, \beta_1, \dots, \beta_n)
\end{align*}
Beim Gradientenverfahren beginnt man mit beliebigen Startwerten $\alpha^0$ und $\beta_k^0$ und einer Schrittweite $s \in \mathbb{R}^+$. Vom Startpunkt aus geht man nun in die Richtung des steilsten Anstieges der Funktion und erhält dadurch neue Werte $\alpha^1$ und $\beta_k^1$. Diese Richtung ist der Gradient der Funktion $L$.

Der Gradient ist ein Vektor, der sich aus den partiellen Ableitungen von $L$ nach jeweils einer Variablen zusammensetzt und wird wie folgt notiert:
\begin{align*}
    \text{grad}(L) = \begin{pmatrix}
        \partial L / \partial \alpha \\
        \partial L / \partial \beta_1 \\
        \vdots \\
        \partial L / \partial \beta_n
    \end{pmatrix}
\end{align*}
Der Gradient von $L$ ist wieder eine Funktion, die Werte $\alpha$ und $\beta_k$ auf einen Vektor der Länge $n+1$ abbildet. Der iterative Schritt des Verfahrens definiert sich wie folgt:
\begin{align*}
    \begin{pmatrix}
        \alpha^{i + 1} \\
        \beta_1^{i + 1} \\
        \vdots \\
        \beta_n^{i + 1}
    \end{pmatrix} = \begin{pmatrix}
        \alpha^i \\
        \beta_1^i \\
        \vdots \\
        \beta_n^i
    \end{pmatrix} + s \cdot \text{grad}(L) (\alpha^i, \beta_1^i, \dots, \beta_n^i)
\end{align*}
Danach muss noch getestet werden, dass $L$ für die neuen Parameter auch wirklich einen größeren Wert annimmt also zuvor. Falls nicht, muss die Schrittweite $s$ verkleinert werden, zum Beispiel um einen festen, zuvor definierten Faktor.
% TODO: Zitiere Schrittweitenstrategien usw aus geeignetem Buch.

\subsection{Gradient bei logistischer Regression}
\label{subsection:2:2:2}

Um das Gradientenverfahren bei logistischer Regression einsetzen zu können, muss der Gradient für den Logarithmus der Likelihoodfunktion bekannt sein. In diesem Kapitel berechnen wir also die partiellen Ableitungen nach allen Parametern.

Um $L_{log} = \ln(L)$ partiell ableiten zu können, berechnen wir zuerst die partiellen Ableitungen aller $\pi_i$. Für die partielle Ableitung nach $\alpha$ ergibt sich mit der Kettenregel folgende Funktion:
\begin{align*}
    \dfrac{\partial \pi_i}{\partial \alpha} &= - \left( 1 + \exp \left(- \alpha - \sum_{j=1}^n \beta_j \cdot x_{i, j} \right) \right)^{-2} \cdot \exp \left(- \alpha - \sum_{j=1}^n \beta_j \cdot x_{i, j} \right) \cdot (-1) \\
    &= \left( 1 + \exp \left(- \alpha - \sum_{j=1}^n \beta_j \cdot x_{i, j} \right) \right)^{-1} \cdot \left( 1 + \exp \left(\alpha + \sum_{j=1}^n \beta_j \cdot x_{i, j} \right) \right)^{-1} \\
    &= \pi_i(\alpha, \beta_1, \dots, \beta_n) \cdot \pi_i(- \alpha, - \beta_1, \dots, - \beta_n)
\end{align*}
Die partiellen Ableitungen einem der $\beta_k$ für $k = 1, \dots, n$ kann fast analog gebildet werden. Bei der Anwendung der Kettenregel auf die innerste lineare Funktion bleibt jedoch noch der konstante Faktor $x_{i, k}$ übrig.
\begin{align*}
    \dfrac{\partial \pi_i}{\partial \beta_k} = x_{i, k} \cdot \pi_i(\alpha, \beta_1, \dots, \beta_n) \cdot \pi_i(- \alpha, - \beta_1, \dots, - \beta_n)
\end{align*}
Bevor wir $L_{log}$ ableiten, definieren wir Hilfsvariablen $\tilde\alpha := (2 y_i - 1) \alpha$ und $\tilde\beta_i := (2 y_i - 1) \beta_i$ für $i = 1, \dots, n$. Damit erhält man:
\begin{align*}
    L_{log}(\alpha, \beta_1, \dots, \beta_n) &= \ln(L(\alpha, \beta_1, \dots, \beta_n)) \\
    &= \sum_{i=1}^m \ln \left(y_i \cdot \pi(\tilde\alpha, \tilde\beta_1, \dots, \tilde\beta_n) \right)
\end{align*}
Leitet man nach $\alpha$ ab, so erhält man:
\begin{align*}
    \dfrac{\partial L_{log}}{\partial \alpha} &= \sum_{i=1}^m \dfrac{\partial}{\partial \alpha} \bigg( \ln \left( \pi_i(\tilde\alpha, \tilde\beta_1, \dots, \tilde\beta_n) \right) \bigg) \\
    &= \sum_{i=1}^m \left( \pi_i(\tilde\alpha, \tilde\beta_1, \dots, \tilde\beta_n) \right)^{-1} \cdot \dfrac{\partial \pi_i}{\partial \tilde\alpha} \cdot \dfrac{\partial \tilde\alpha}{\partial \alpha} \\
    &= \sum_{i=1}^m \left( \pi_i(\tilde\alpha, \tilde\beta_1, \dots, \tilde\beta_n) \right)^{-1} \cdot \pi_i(\tilde\alpha, \tilde\beta_1, \dots, \tilde\beta_n) \cdot \big(1 - \pi_i(\tilde\alpha, \tilde\beta_1, \dots, \tilde\beta_n) \big) \cdot (2 y_i - 1) \\
    &= \sum_{i=1}^m \big(1 - \pi_i(\tilde\alpha, \tilde\beta_1, \dots, \tilde\beta_n) \big) \cdot (2 y_i - 1)
\end{align*}
Für die partielle Ableitung nach $\beta_k$ erhält man analog:
\begin{align*}
    \dfrac{\partial L_{log}}{\partial \beta_k} = \sum_{i=1}^m x_{i, k} \cdot \big(1 - \pi_i(\tilde\alpha, \tilde\beta_1, \dots, \tilde\beta_n) \big) \cdot (2 y_i - 1)
\end{align*}
Wir betrachten nun die einzelnen Summanden der partiellen Ableitungen getrennt für die beiden möglichen Werten von $y_i$. Ist $y_i = 0$ dann gilt:
\begin{align*}
    \big(1 - \pi_i(\tilde\alpha, \tilde\beta_1, \dots, \tilde\beta_n) \big) \cdot (2 y_i - 1) &= \big(1 - \pi_i(- \alpha, - \beta_1, \dots, - \beta_n) \big) \cdot (- 1) \\
    &= - 1 + \big(1 - \pi_i(\alpha, \beta_1, \dots, \beta_n) \big) \\
    &= - \pi_i(\alpha, \beta_1, \dots, \beta_n)
\end{align*}
Für $y_i = 1$ ergibt sich folgendes:
\begin{align*}
    \big(1 - \pi_i(\tilde\alpha, \tilde\beta_1, \dots, \tilde\beta_n) \big) \cdot (2 y_i - 1) &= \big(1 - \pi_i(\alpha, \beta_1, \dots, \beta_n) \big) \cdot (2 - 1) \\
    &= 1 - \pi_i(\alpha, \beta_1, \dots, \beta_n)
\end{align*}
Damit können wir die partiellen Ableitungen weiter vereinfachen:
\begin{align*}
    \dfrac{\partial L_{log}}{\partial \alpha} &= \sum_{i=1}^m y_i - \pi_i(\alpha, \beta_1, \dots, \beta_n) \\
    \dfrac{\partial L_{log}}{\partial \beta_k} &= \sum_{i=1}^m x_{i, k} \cdot \big(y_i - \pi_i(\alpha, \beta_1, \dots, \beta_n) \big)
\end{align*}
Diese Darstellung der partiellen Ableitungen verwenden wir später in SQL zur Berechnung des Gradienten. Der Term innerhalb der Summe wird für jedes Tupel der Relation der Datenpunkte berechnet, danach wird das resultierende Attribut mit einer Gruppierung summiert.
