\chapter{Grundlagen statistischer Methoden}

Bei der Regressionsanalyse geht es allgemein darum, das Verhalten einer Größe $Y$ in Abhängigkeit einer oder mehrerer anderer Größen $X_1, X_2, \dots, X_n$ zu prognostizieren. Die Größe $Y$ wird abhängig genannt, die Größen $X_i$ nennt man unabhängig. Die Prognose entspricht einem funktionalen Zusammenhang der folgenden Art:
\begin{align*}
    Y = f(X_1, \dots, X_n)
\end{align*}
Als Grundlage für den Findungsprozess dieser Funktion $f$ besitzt man eine Menge von $N$ Datenpunkten:
\begin{align*}
    (x_{1, j}, x_{2, j}, \dots, x_{n, j}, y_j) ~~~ \text{für} ~~ j = 1, \dots, N
\end{align*}
Im Allgemeinen ist es nicht möglich eine einfache Funktion $f$ zu finden, die alle Datenpunkte exakt prognostiziert. Stattdessen über- und unterschätzt die Funktion $f$ die realen Datenpunkte. Die konkrete Abweichung für den $j$-ten Datenpunkt wird in der Satistik als Fehler bezeichnet:
\begin{align*}
    e_j = y_j - f(x_{1, j}, \dots, x_{n, j})
\end{align*}

\section{Lineare Regression}

Bei der linearen Regression geht man von einem linearen Zusammenhang zwischen der abhängigen und den unabhängigen Variablen aus. Die Funktion $f$ ist also von folgender Form:
\begin{align*}
    f(x_1, \dots, x_n) = \beta_0 + \sum_{i=1}^n \beta_i \cdot x_i ~~~ \text{mit} ~~ \beta_i \in \mathbb{R}
\end{align*}
Man möchte diejenigen Parameter $\beta_i$ bestimmen, so dass die entstehende Funktion die vorgegebenen Datenpunkte möglichst gut annähert. Das Maß für die Qualität einer Funktion $f$ definiert durch die Parameter $\beta_0, \dots, \beta_n$ ist die Summe der quadrierten Fehlerterme:
\begin{align*}
    E(\beta_0, \dots, \beta_n) = \sum_{j=1}^N e_j^2 = \sum_{j=1}^N \left( y_j - \beta_0 - \sum_{i=1}^n \beta_i \cdot x_{i, j} \right)^2
\end{align*}
Ziel ist es nun die Parameter so zu wählen, dass die Funktion $E$ miminiert wird. Dieses Vorgehen ist als Methode der kleinsten Quadrate bekannt.
