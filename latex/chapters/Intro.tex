\chapter{Einführung und typische statistische Problemstellungen}
\label{chapter:1}

Die Statistik ist ein Teilgebiet der Mathematik, in welchem Methoden zum Umgang und zur Verarbeitung von Daten behandelt werden. Dabei wird oft ein vorhandener Satz an Daten, auch Stichprobe genannt, betrachtet und analysiert, um daraus Vorhersagen für die Gesamtheit aller Daten zu treffen. Den Teilbereich der Statistik, welcher sich mit solchen Problemen befasst, nennt man induktive oder schließende Statistik.

Betrachtet man beispielsweise die Körpergröße und das Gewicht von 100 Testpersonen, dann kann man sich fragen, ob diese beiden Merkmale in Zusammenhang stehen. Insbesondere ist interessant, wie man einen möglichen Zusammenhang quantitativ darstellen kann und ob man neben der Körpergröße auch andere Faktoren für das Gewicht einer Person betrachten sollte. Das sind beispielhafte Typen von Fragen, welchen man in der Statistik oft begegnet.

Die Statistik zeigt Methoden und Vorgehensweisen auf, wie man solche Fragen angehen und beantworten kann. Ein oft verwendetes Verfahren ist die Regression bzw. die Regressionsanalyse. Hier sucht und analysiert man Beziehungen zwischen mehreren Variablen und versucht diese quantitativ zu beschreiben.

Greifen wir das obige Beispiel wieder auf: Bei der Regression sucht man nach einer Formel, welche für gegebene Körpergröße das Gewicht einer Person möglichst gut schätzt. Oft trifft man Annahmen über die Art der Beziehung zwischen den Dimensionen, um die Suche a priori einzugrenzen. Man beschränkt sich in vielen Fällen auf lineare Funktionen, da solche leicht zu behandeln sind. In der Praxis sind aber auch allgemeine Potenzfunktionen, exponentielle Funktionen oder logistische Funktionen bzw. Beziehungen oft anzutreffen.

Es gibt speziell für Statistik entwickelte Software, seien es einfache Programmiersprachen wie das R-Projekt oder komplexere Programme mit grafischem Interface wie SPSS von IBM. Die Daten, welche man als Basis für Analysen verwendet, müssen aber an einer anderen Stelle gespeichert und verwaltet werden. Oft liegen diese in einer Datenbank und müssen zuerst in das Analysetool importiert werden.

Konzeptuell ist eine Datenbank nicht für Regressionsanalyse geschaffen. Dennoch ist es in relationalen Datenbanksystemen mit SQL möglich, Regression direkt in der Datenbank durchzuführen. Damit übergeht man den eben genannten Schritt des Importierens. Man kann außerdem die Ergebnisse der Analyse direkt aus der Datenbank abfragen oder dort weiterverwenden.

Diese Arbeit wird zuerst das Konzept der Regression konkreter einführen und die mathematischen Grundlagen darlegen. Darauf aufbauend betrachten wir Implementierungen für Regressionsanalyse in verschiedenen Programmiersprachen. Hier soll insbesondere die Anwendung von Regressionsanalyse mit Hilfe von SQL demonstriert werden. Danach wollen wir die Sprachen bezüglich der Laufzeit miteinander vergleichen und noch kurz auf das Erweiterungspotenzial für relationale Datenbanksysteme eingehen.
