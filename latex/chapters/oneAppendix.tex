% TODO: Skripte ganz zum Schluss nochmal pasten
\chapter{Python-Skript zum Generieren der Beispieldaten}
\label{appendix:A}

\begin{minted}[linenos,breaklines,frame=single]{python}
  import sys
  import random
  import math

  # Erzeuge Funktion, die die Daten in eine csv-Datei schreibt.
  def outputCsv(data):
    # Erzeuge ein Array mit den zu schreibenden Zeilen und übergebe die Spaltennamen.
    output = ["%s,%s,%s,%s" % (
      "age",
      "purchases",
      "money",
      "premium"
    )]

    # Iteriere über alle Datenpukte.
    for datapoint in data:
      # Hänge eine Zeile an das output-Array an.
      output.append("%s,%s,%s,%a" % (
        datapoint["age"],
        datapoint["purchases"],
        datapoint["money"],
        datapoint["premium"]
      ))

    # Öffne eine csv-Datei.
    f = open("sample.csv", "w")

    # Schreibe das output-Array als mit Zeilenumbrüchen gejointen String in die Datei.
    f.write("\n".join(output))

    return

  # Erzeuge Funktion, die die Daten in eine sql-Datei schreibt.
  def outputSql(data):
    # Erzeuge ein Array mit den zu schreibenden Zeilen.
    # Füge SQL-Abfragen ein, die eine eventuell bestehende Tabelle löscht und neu erstellet.
    # Beginne mit der INSERT-Abfrage.
    output = [
      "DROP TABLE IF EXISTS sample;",
      "",
      "CREATE TABLE sample (",
      "  age INTEGER,",
      "  purchases INTEGER,",
      "  money INTEGER,",
      "  premium INTEGER",
      ");",
      "",
      "INSERT INTO sample (%s,%s,%s,%s) VALUES" % (
        "age",
        "purchases",
        "money",
        "premium"
      )
    ]

    # Iteriere über alle Datenpunkte.
    for datapoint in data:
      # Füge eine Zeile an die INSERT-Abfrage an.
      output.append("(%s,%s,%s,%s)," % (
        datapoint["age"],
        datapoint["purchases"],
        datapoint["money"],
        datapoint["premium"]
      ))

    # Ersetze das letzte Komma durch ein Semikolon, um die INSERT-Abfrage zu beenden.
    output[-1] = output[-1][:-1] + ";"

    # Öffne eine sql-Datei.
    f = open("sample.sql", "w")

    # Schreibe das output-Array als mit Zeilenumbrüchen gejointen String in die Datei.
    f.write("\n".join(output))

    return

  def main(argv):
    # Beende die Ausführung, wenn die Anzahl der zu erzeugenden Datenpunkte nicht übergeben wurde.
    if len(argv) < 2:
      print("Please provide number of datapoints that shall be generated.")
      return

    # Erzeuge ein leeres Array für die Daten.
    data = []

    # Iteriere über die Anzahl der zu erzeugenden Datenpunkte.
    for i in range(0, int(argv[1])):
      # Bestimme pseudo-zufällige Werte für den Datenpunkt.
      age = int(max(random.normalvariate(25, 10) + 10, 18))
      purchases = int(max(random.normalvariate(10, 10), 1))
      money = int(max(purchases * 25 + random.normalvariate(0, (math.log(purchases) + 1) * 12), 0.01) * 100)
      if random.uniform(0, 1) > math.exp(0.2 * purchases - 2) / (1 + math.exp(0.2 * purchases - 2)):
        premium = 0
      else:
        premium = 1

      # Hänge den Datenpunkt an das Array an.
      data.append({
        "age": age,
        "purchases": purchases,
        "money": money,
        "premium": premium
      })

    # Schreibe die generierten Daten in eine .csv und eine .sql Datei.
    outputCsv(data)
    outputSql(data)
    return

  # Führe die main-Funktion aus.
  if __name__ == "__main__":
    main(sys.argv)
\end{minted}

\chapter{R-Skripte}
\label{appendix:B}

\section{Einfache lineare Regression}
\label{appendix:B:1}

\begin{minted}[linenos,breaklines,frame=single]{r}
  # Lese die übergebenen Argumente ein.
  args = commandArgs(trailingOnly = TRUE)

  # Setzte default-Werte für die Anzahl der Datenpunkte und ob geplottet werden soll.
  n <- 1000
  plot <- TRUE

  # Ändere die default-Werte, falls entsprechende Argumente übergeben wurden.
  if (length(args) == 1) {
    if (substr(args[1], 1, 1) == "-") {
      plot <- FALSE
    } else {
      n = strtoi(args[1])
    }
  }
  if (length(args) == 2) {
    if (substr(args[1], 1, 1) == "-") {
      n = strtoi(args[2])
    } else {
      n = strtoi(args[1])
    }
    plot <- FALSE
  }

  # Speichere die aktuelle Zeit zur Zeitmessung.
  start_time <- Sys.time()

  # Lies die Daten aus der csv-Datei ein.
  data <- head(read.csv2("./data/sample.csv", sep = ",", header = TRUE), n)

  # Definiere das Modell.
  modell <- as.formula("money ~ purchases")

  # Führe die Regression durch.
  slr <- lm(modell, data = data)

  # Speichere die aktuelle Zeit zur Zeitmessung.
  end_time <- Sys.time()

  # Drucke die Ergebnisse der Regressionsanalyse und die Laufzeit.
  print(slr)
  print(end_time - start_time)

  # Erstelle einen Plot.
  if (plot) {
    # Bestimme die Grenzen für die unabhängige Variable.
    xmin <- min(data$purchases)
    xmax <- max(data$purchases)

    # Bestimme die Grenzen für die abhängige Variable.
    ymin <- min(data$money)
    ymax <- max(data$money)

    # Bestimme die Koeffizienten aus der Regressionsanalyse.
    b0 <- coef(slr["coefficients"])[1]
    b1 <- coef(slr["coefficients"])[2]

    # Erzeuge Vektoren zum Plot der linearen Funktion.
    xplot <- c(xmin - 1, xmax + 1)
    yplot <- c(b0 + (xmin - 1) * b1, b0 + (xmax + 1) * b1)

    # Erstelle den Plot.
    plot(
      c(xmin - 1, xmax + 1),
      c(ymin - 1, ymax + 1),
      type = "n",
      xlab = "purchases",
      ylab = "money",
      main = "Einfache lineare Regression",
      sub = paste("money = ", b0, "+", b1, "* purchases"),
      col.sub = "darkgray"
    )
    # Füge die Datenpunkte ein.
    lines(
      data$purchases,
      data$money,
      type="p"
    )
    # Füge die Ausgleichsgerade ein.
    lines(
      xplot,
      yplot,
      col = "red",
      lwd = 2
    )
  }
\end{minted}

\section{Multiple lineare Regression}
\label{appendix:B:2}

\begin{minted}[linenos,breaklines,frame=single]{r}
  # Lese die übergebenen Argumente ein.
  args = commandArgs(trailingOnly = TRUE)

  # Setzte default-Wert für die Anzahl der Datenpunkte.
  n = 1000

  # Ändere den default-Wert, falls ein entsprechendes Argumente übergeben wurde.
  if (length(args) > 0) {
    n = strtoi(args[1])
  }

  # Speichere die aktuelle Zeit zur Zeitmessung.
  start_time <- Sys.time()

  # Lies die Daten aus der csv-Datei ein.
  data <- head(read.csv2("./data/sample.csv", sep = ",", header = TRUE), n)

  # Definiere das Modell.
  modell <- as.formula("money ~ purchases + age")

  # Führe die Regression durch.
  mlr <- lm(modell, data = data)

  # Speichere die aktuelle Zeit zur Zeitmessung.
  end_time <- Sys.time()

  # Drucke die Ergebnisse der Regressionsanalyse und die Laufzeit.
  print(mlr)
  print(end_time - start_time)
\end{minted}

\section{Logistische Regression}
\label{appendix:B:3}

\begin{minted}[linenos,breaklines,frame=single]{r}
  # Lese die übergebenen Argumente ein.
  args = commandArgs(trailingOnly = TRUE)

  # Setzte default-Werte für die Anzahl der Datenpunkte und ob geplottet werden soll.
  n <- 1000
  plot <- TRUE

  # Ändere die default-Werte, falls entsprechende Argumente übergeben wurden.
  if (length(args) == 1) {
    if (substr(args[1], 1, 1) == "-") {
      plot <- FALSE
    } else {
      n = strtoi(args[1])
    }
  }
  if (length(args) == 2) {
    if (substr(args[1], 1, 1) == "-") {
      n = strtoi(args[2])
    } else {
      n = strtoi(args[1])
    }
    plot <- FALSE
  }

  # Speichere die aktuelle Zeit zur Zeitmessung.
  start_time <- Sys.time()

  # Lies die Daten aus der csv-Datei ein.
  data <- head(read.csv2("./data/sample.csv", sep = ",", header = TRUE), n)

  # Definiere das Modell.
  modell <- as.formula("premium ~ money")

  # Führe die Regression durch.
  logit <- glm(modell, family = binomial, data = data)

  # Speichere die aktuelle Zeit zur Zeitmessung.
  end_time <- Sys.time()

  # Drucke die Ergebnisse der Regressionsanalyse und die Laufzeit.
  print(logit)
  print(end_time - start_time)

  # Erstelle einen Plot.
  if (plot) {
    # Bestimme die Grenzen für die unabhängige Variable.
    xmin <- min(data$money)
    xmax <- max(data$money)

    # Bestimme eine Funktion, die den Wert der logistischen Funktion berechnet.
    logitFunction <- function(x){
      # Verwende die Parameter aus der Regressionsanalyse.
      b0 <- coef(logit["coefficients"])[1]
      b1 <- coef(logit["coefficients"])[2]
      c <- b0 + x * b1
      return(exp(c) / (1 + exp(c)))
    }

    # Erzeuge Vektoren zum Plot der logistischen Funktion.
    xplot <- seq(xmin - 1, xmax + 1, 1000)
    yplot <- logitFunction(xplot)

    # Erstelle den Plot.
    plot(
      c(xmin - 1, xmax + 1),
      c(-0.2, 1.2),
      type = "n",
      xlab = "money",
      ylab = "premium",
      main = "Logistische Regression"
    )
    # Füge die Datenpunkte ein.
    lines(
      data$money,
      data$premium,
      type="p"
    )
    # Füge die logistische Funktion ein.
    lines(
      xplot,
      yplot,
      col = "red",
      lwd = 2
    )
  }
\end{minted}

\chapter{TensorFlow-Skripte}
\label{appendix:C}

\section{Einfache lineare Regression}
\label{appendix:C:1}

\begin{minted}[linenos,breaklines,frame=single]{python}
  import numpy as np
  import tensorflow as tf
  import os.path as p
  import csv
  import sys
  import matplotlib.pyplot as plt
  from time import time

  # Erstelle eine Funktion zum einlesen der Daten aus der csv-Daten.
  def get_data(n_samples):
    # Bestimme den Pfad der csv-Datei und öffne die Datei.
    filename = p.abspath(p.join(p.dirname(p.realpath(__file__)), "..", "data", "sample.csv"))
    csvfile = open(filename, newline="")
    csvreader = csv.reader(csvfile, delimiter=",", quotechar="|")

    # Erstelle leere Arrays für die Daten.
    x = []
    x_plot = []
    y = []

    # Iteriere über die Zeilen der csv-Datei.
    for row in csvreader:
      # Überspringe die erste Zeile.
      if not row[0] == "age":
        # Füge die Daten der Zeile zum jeweiligen Array hinzu.
        x.append([int(row[1])])
        x_plot.append(int(row[1]))
        y.append(int(row[2]))

    # Gib die Arrays bis zu der gewünschten Menge an Datenpunkten zurück.
    return (np.array(x[:n_samples]), x_plot[:n_samples], np.transpose([y[:n_samples]]))

  def main(argv):
    # Bestimme default-Werte für die Anzahl der Datenpunkte und ob geplottet werden soll.
    datapoint_size = 1000
    plot = True

    # Ändere die default-Werte, wenn entsprechende Argumente übergeben wurden.
    if len(argv) == 2:
      if argv[1] == "-":
        plot = False
      else:
        datapoint_size = int(argv[1])
    elif len(argv) == 3:
      plot = False
      if argv[1] == "-":
        datapoint_size = int(argv[2])
      else:
        datapoint_size = int(argv[1])

    # Bestimme die aktuelle Zeit zur Zeitmessung.
    start_time = time()

    # Bestimme die Anzahl der Iterationen und die Schrittweite (anhängig von der Anzahl der Datenpunkte.)
    steps = 2000
    if datapoint_size <= 10:
      learn_rate = 0.0076
    elif datapoint_size <= 100:
      learn_rate = 0.0064
    elif datapoint_size <= 1000:
      learn_rate = 0.0056
    elif datapoint_size <= 10000:
      learn_rate = 0.0054
    elif datapoint_size <= 100000:
      learn_rate = 0.0054

    # Deklariere die Platzhalter und Variablen.
    x = tf.placeholder(tf.float32, [None, 1])
    y = tf.placeholder(tf.float32, [None, 1])
    alpha = tf.Variable(tf.zeros([1]))
    beta = tf.Variable(tf.zeros([1, 1]))
    y_calc = tf.matmul(x, beta) + alpha

    # Definiere die Kostenfunktion und die Minimierungsoperation.
    cost = tf.reduce_mean(tf.square(y - y_calc))
    train_step = tf.train.GradientDescentOptimizer(learn_rate).minimize(cost)

    # Importiere die Daten.
    (all_xs, plot_xs, all_ys) = get_data(datapoint_size)

    # Starte eine Session in TensorFlow.
    sess = tf.Session()
    init = tf.global_variables_initializer()
    sess.run(init)

    # Iteriere und trainiere.
    for i in range(steps):
      feed = { x: all_xs, y: all_ys }
      sess.run(train_step, feed_dict=feed)

    # Bestimme die aktuellen Parameterwerte nach der Anzahl der Iterationen.
    (curr_alpha, curr_beta, curr_cost) = sess.run([alpha, beta, cost], feed_dict=feed)

    # Bestimme die aktuelle Zeit zur Zeitmessung.
    end_time = time()

    # Drucke die Ergebnisse.
    print("alpha:  %f" % curr_alpha)
    print("beta:   %f" % curr_beta)
    print("cost:   %f" % curr_cost)
    print("")
    print("time elapsed:  %f sec" % (end_time - start_time))

    # Erstelle einen Plot (falls gewünscht).
    if plot:
      plt.plot(plot_xs, all_ys, "ro", label="Original data")
      plt.plot(plot_xs, curr_beta * all_xs + curr_alpha , label="Fitted line")
      plt.legend()
      plt.show()

  # Führe die main-Funktion aus.
  if __name__ == "__main__":
    main(sys.argv)
\end{minted}

\section{Multiple lineare Regression}
\label{appendix:C:2}

\begin{minted}[linenos,breaklines,frame=single]{python}
  import numpy as np
  import tensorflow as tf
  import os.path as p
  import csv
  import sys
  from time import time

  # Erstelle eine Funktion zum einlesen der Daten aus der csv-Daten.
  def get_data(n_samples):
    # Bestimme den Pfad der csv-Datei und öffne die Datei.
    filename = p.abspath(p.join(p.dirname(p.realpath(__file__)), "..", "data", "sample.csv"))
    csvfile = open(filename, newline="")
    csvreader = csv.reader(csvfile, delimiter=",", quotechar="|")

    # Erstelle leere Arrays für die Daten.
    x = []
    y = []

    # Iteriere über die Zeilen der csv-Datei.
    for row in csvreader:
      # Überspringe die erste Zeile.
      if not row[0] == "age":
        # Füge die Daten der Zeile zum jeweiligen Array hinzu.
        x.append([int(row[1]), int(row[0])])
        y.append(int(row[2]))

    # Gib die Arrays bis zu der gewünschten Menge an Datenpunkten zurück.
    return (np.array(x[:n_samples]), np.transpose([y[:n_samples]]))

  def main(argv):
    # Bestimme default-Wert für die Anzahl der Datenpunkte.
    datapoint_size = 1000

    # Ändere den default-Wert, wenn ein entsprechendes Argument übergeben wurde.
    if len(argv) == 2:
      datapoint_size = int(argv[1])

    # Bestimme die aktuelle Zeit zur Zeitmessung.
    start_time = time()

    # Bestimme die Anzahl der Iterationen und die Schrittweite (anhängig von der Anzahl der Datenpunkte.)
    steps = 50000
    if datapoint_size <= 10:
      learn_rate = 0.00093
    elif datapoint_size <= 100:
      learn_rate = 0.00078
    elif datapoint_size <= 1000:
      learn_rate = 0.0007
    elif datapoint_size <= 10000:
      learn_rate = 0.00071
    elif datapoint_size <= 100000:
      learn_rate = 0.00071

    # Deklariere die Platzhalter und Variablen.
    x = tf.placeholder(tf.float32, [None, 2])
    y = tf.placeholder(tf.float32, [None, 1])
    alpha = tf.Variable(tf.zeros([1]))
    beta = tf.Variable(tf.zeros([2, 1]))
    y_calc = tf.matmul(x, beta) + alpha

    # Definiere die Kostenfunktion und die Minimierungsoperation.
    cost = tf.reduce_mean(tf.square(y - y_calc))
    train_step = tf.train.GradientDescentOptimizer(learn_rate).minimize(cost)

    # Importiere die Daten.
    (all_xs, all_ys) = get_data(datapoint_size)

    # Starte eine Session in TensorFlow.
    sess = tf.Session()
    init = tf.global_variables_initializer()
    sess.run(init)

    # Iteriere und trainiere.
    for i in range(steps):
      feed = { x: all_xs, y: all_ys }
      sess.run(train_step, feed_dict=feed)

    # Bestimme die aktuellen Parameterwerte nach der Anzahl der Iterationen.
    (curr_alpha, curr_beta, curr_cost) = sess.run([alpha, beta, cost], feed_dict=feed)

    # Bestimme die aktuelle Zeit zur Zeitmessung.
    end_time = time()

    # Drucke die Ergebnisse.
    print("alpha:           %f" % curr_alpha)
    print("beta_purchases:  %f" % curr_beta[0])
    print("beta_age:        %f" % curr_beta[1])
    print("cost:            %f" % curr_cost)
    print("")
    print("time elapsed: %f sec" % (end_time - start_time))

  # Führe die main-Funktion aus.
  if __name__ == "__main__":
    main(sys.argv)
\end{minted}

\section{Logistische Regression}
\label{appendix:C:3}

\begin{minted}[linenos,breaklines,frame=single]{python}
  import numpy as np
  import tensorflow as tf
  import os.path as p
  import csv
  import sys
  import matplotlib.pyplot as plt
  from time import time

  # Erstelle eine Funktion zum einlesen der Daten aus der csv-Daten.
  def get_data(n_samples):
    # Bestimme den Pfad der csv-Datei und öffne die Datei.
    filename = p.abspath(p.join(p.dirname(p.realpath(__file__)), "..", "data", "sample.csv"))
    csvfile = open(filename, newline="")
    csvreader = csv.reader(csvfile, delimiter=",", quotechar="|")

    # Erstelle leere Arrays für die Daten.
    x = []
    x_plot = []
    y = []

    # Iteriere über die Zeilen der csv-Datei.
    for row in csvreader:
      # Überspringe die erste Zeile.
      if not row[0] == "age":
        # Füge die Daten der Zeile zum jeweiligen Array hinzu.
        x.append([int(row[2])])
        x_plot.append(int(row[2]))
        y.append(int(row[3]))

    # Gib die Arrays bis zu der gewünschten Menge an Datenpunkten zurück.
    return (np.array(x[:n_samples]), x_plot[:n_samples], np.transpose([y[:n_samples]]))

  def main(argv):
    # Bestimme default-Werte für die Anzahl der Datenpunkte und ob geplottet werden soll.
    datapoint_size = 1000
    plot = True

    # Ändere die default-Werte, wenn entsprechende Argumente übergeben wurden.
    if len(argv) == 2:
      if argv[1] == "-":
        plot = False
      else:
        datapoint_size = int(argv[1])
    elif len(argv) == 3:
      plot = False
      if argv[1] == "-":
        datapoint_size = int(argv[2])
      else:
        datapoint_size = int(argv[1])

    # Bestimme die aktuelle Zeit zur Zeitmessung.
    start_time = time()

    # Bestimme die Anzahl der Iterationen und die Schrittweite (anhängig von der Anzahl der Datenpunkte.)
    steps = 1000
    if datapoint_size == 10:
      learn_rate = 1
    elif datapoint_size == 100:
      learn_rate = 0.1
    elif datapoint_size == 1000:
      learn_rate = 0.01
    elif datapoint_size == 10000:
      learn_rate = 0.001
    elif datapoint_size == 100000:
      learn_rate = 0.0001

    # Deklariere die Platzhalter und Variablen.
    x = tf.placeholder(tf.float32, [None, 1])
    y = tf.placeholder(tf.float32, [None, 1])
    alpha = tf.Variable(tf.zeros([1]))
    beta = tf.Variable(tf.zeros([1, 1]))
    y_calc = 1 / (1 + tf.exp(- tf.matmul(x, beta) - alpha))

    # Definiere die Kostenfunktion und die Minimierungsoperation.
    cost = - tf.reduce_sum(
      tf.log(
        y * y_calc +
        (1 - y) * (1 - y_calc)
      )
    )
    train_step = tf.train.GradientDescentOptimizer(learn_rate).minimize(cost)

    # Importiere die Daten.
    (all_xs, plot_xs, all_ys) = get_data(datapoint_size)

    # Transformiere die unabhägige Variable linear ins Interval zwischen 0 und 1.
    min_x = min(all_xs)
    max_x = max(all_xs)
    all_xs = (all_xs - min_x) / (max_x - min_x)

    # Starte eine Session in TensorFlow.
    sess = tf.Session()
    init = tf.global_variables_initializer()
    sess.run(init)

    # Iteriere und trainiere.
    for i in range(steps):
      feed = { x: all_xs, y: all_ys }
      sess.run(train_step, feed_dict=feed)

    # Bestimme die aktuellen Parameterwerte nach der Anzahl der Iterationen.
    (curr_alpha, curr_beta, curr_cost) = sess.run([alpha, beta, cost], feed_dict=feed)

    # Transformiere die Parameter linear, um den originalen Daten zu entsprechen.
    curr_beta = curr_beta / (max_x - min_x)
    curr_alpha = curr_alpha - curr_beta * min_x

    # Bestimme die aktuelle Zeit zur Zeitmessung.
    end_time = time()

    # Drucke die Ergebnisse.
    print("alpha:  %f" % curr_alpha)
    print("beta:   %f" % curr_beta)
    print("cost:   %f" % curr_cost)
    print("")
    print("time elapsed: %f sec" % (end_time - start_time))

    # Erstelle einen Plot (falls gewünscht).
    if plot:
      all_xs = all_xs * (max_x - min_x) + min_x
      plot_ys = 1 / (1 + np.exp(- curr_beta * all_xs - curr_alpha))
      plot_order = np.argsort(plot_xs)
      plt.plot(plot_xs, all_ys, "ro", label="Original data")
      plt.plot(np.array(plot_xs)[plot_order], np.array(plot_ys)[plot_order], label="Fitted line")
      plt.legend()
      plt.show()

  # Führe die main-Funktion aus.
  if __name__ == "__main__":
    main(sys.argv)
\end{minted}

\chapter{MySQL-Skripte}
\label{appendix:D}

\section{Einfache lineare Regression}
\label{appendix:D:1}

\begin{minted}[linenos,breaklines,frame=single]{sql}
  -- Lösche die bestehende Prozedur, falls vorhanden.
  DROP PROCEDURE IF EXISTS simple_linear_regression;

  DELIMITER ;;

  -- Erstelle die Prozedur für einfache lineare Regression.
  CREATE PROCEDURE `simple_linear_regression`(IN number_datapoints INT(11))
  BEGIN

  -- Deklariere die verwendeten Variablen.
  DECLARE purchases_mean DECIMAL(65, 30);
  DECLARE money_mean DECIMAL(65, 30);
  DECLARE alpha DECIMAL(65, 30);
  DECLARE beta DECIMAL(65, 30);

  -- Erstelle eine temporäre Relation für die zu verwendenden Datenpunkte.
  DROP TEMPORARY TABLE IF EXISTS datapoints;
  CREATE TEMPORARY TABLE datapoints (
    purchases INT(11),
    money INT(11)
  );

  -- Füge die gewünschte Anzahl der Datenpunkte in die temporäre Relation ein.
  INSERT INTO datapoints
  SELECT purchases, money
  FROM sample
  LIMIT number_datapoints;

  -- Berechne die Mittelwerte der abhängigen und unabhängigen Variable.
  SET purchases_mean = (
    SELECT AVG(purchases)
    FROM datapoints
  );
  SET money_mean = (
    SELECT AVG(money)
    FROM datapoints
  );

  -- Berechne beta.
  SET beta = (
    SELECT SUM((purchases - purchases_mean) * (money - money_mean))
    FROM datapoints
  );
  SET beta = beta / (
    SELECT SUM(POWER(purchases - purchases_mean, 2))
    FROM datapoints
  );

  -- Berechne alpha.
  SET alpha = money_mean - (beta * purchases_mean);

  -- Gib eine Relation mit Parametername und zugehörigem Wert zurück.
  SELECT 'alpha' AS `variable`, alpha AS `value`
  UNION
  SELECT 'beta' AS `variable`, beta AS `value`;

  -- Lösche die temporäre Relation mit den Datenpunkten wieder.
  DROP TEMPORARY TABLE IF EXISTS datapoints;

  END;;

  DELIMITER ;
\end{minted}

\section{Multiple lineare Regression}
\label{appendix:D:2}

\begin{minted}[linenos,breaklines,frame=single]{sql}
  -- Lösche die bestehende Prozedur, falls vorhanden.
  DROP PROCEDURE IF EXISTS multiple_linear_regression;

  DELIMITER ;;

  -- Erstelle die Prozedur für multiple lineare Regression.
  CREATE PROCEDURE multiple_linear_regression(IN number_datapoints INT(11))
  BEGIN

  -- Deklariere die verwendeten Variablen.
  DECLARE m INT(11);
  DECLARE n INT(11);
  DECLARE counter_1 INT(11);
  DECLARE counter_2 INT(11);
  DECLARE counter_3 INT(11);
  DECLARE pivot DECIMAL(65, 30);

  -- Bestimme die Dimensionsn für die Matrix X.
  SET m = number_datapoints;
  SET n = 3;

  -- Lösche vorhandene temporäre Relationen.
  DROP TEMPORARY TABLE IF EXISTS matrix_X;
  DROP TEMPORARY TABLE IF EXISTS matrix_transposed;
  DROP TEMPORARY TABLE IF EXISTS matrix_product_1;
  DROP TEMPORARY TABLE IF EXISTS matrix_inverse;
  DROP TEMPORARY TABLE IF EXISTS matrix_product_2;
  DROP TEMPORARY TABLE IF EXISTS matrix_y;
  DROP TEMPORARY TABLE IF EXISTS matrix_result;

  -- Erstelle temporäre Relationen für die zu berechnenden Matrizen.
  CREATE TEMPORARY TABLE matrix_X (
    `row` INT(11),
    `column` INT(11),
    `value` DECIMAL(65, 30)
  );
  CREATE TEMPORARY TABLE matrix_transposed (
    `row` INT(11),
    `column` INT(11),
    `value` DECIMAL(65, 30)
  );
  CREATE TEMPORARY TABLE matrix_product_1 (
    `row` INT(11),
    `column` INT(11),
    `value` DECIMAL(65, 30)
  );
  CREATE TEMPORARY TABLE matrix_inverse (
    `row` INT(11),
    `column` INT(11),
    `value` DECIMAL(65, 30)
  );
  CREATE TEMPORARY TABLE matrix_product_2 (
    `row` INT(11),
    `column` INT(11),
    `value` DECIMAL(65, 30)
  );
  CREATE TEMPORARY TABLE matrix_y (
    `row` INT(11),
    `column` INT(11),
    `value` DECIMAL(65, 30)
  );
  CREATE TEMPORARY TABLE matrix_result (
    `row` INT(11),
    `column` INT(11),
    `value` DECIMAL(65, 30)
  );

  -- Füge Werte der unabhängigen Variablen in die Relation matrix_X ein.
  SET @id = 0;

  INSERT INTO matrix_X
  SELECT
    @id := (@id + 1) AS `row`,
    1 AS `column`,
    1 AS `value`
  FROM sample
  LIMIT number_datapoints;

  SET @id = 0;

  INSERT INTO matrix_X
  SELECT
    @id := (@id + 1) AS `row`,
    2 AS `column`,
    purchases AS `value`
  FROM sample
  LIMIT number_datapoints;

  SET @id = 0;

  INSERT INTO matrix_X
  SELECT
    @id := (@id + 1) AS `row`,
    3 AS `column`,
    age AS `value`
  FROM sample
  LIMIT number_datapoints;

  -- Füge Werte der abhängigen Variable in die Relation matrix_y ein.
  SET @id = 0;

  INSERT INTO matrix_y
  SELECT
    @id := (@id + 1) AS `row`,
    1 AS `column`,
    money AS `value`
  FROM sample
  LIMIT number_datapoints;

  -- Berechne matrix_transposed.
  INSERT INTO matrix_transposed
  SELECT
    `column` AS `row`,
    `row` AS `column`,
    `value` AS `value`
  FROM matrix_X;

  -- Berechne matrix_product_1. Iteriere dazu über alle Zeilen und Spalten der Ergebnismatrix.
  SET counter_1 = 1;

  WHILE counter_1 <= n DO

    SET counter_2 = 1;

    WHILE counter_2 <= n DO

      -- Berechne den Wert des aktuellen Matrixelements.
      INSERT INTO matrix_product_1 VALUES (
        counter_1,
        counter_2,
        (
          SELECT SUM(matrix_X.`value` * matrix_transposed.`value`)
          FROM matrix_X, matrix_transposed
          WHERE matrix_X.`column` = counter_2
            AND matrix_transposed.`row` = counter_1
            AND matrix_transposed.`column` = matrix_X.`row`
        )
      );

      SET counter_2 = counter_2 + 1;

    END WHILE;

    SET counter_1 = counter_1 + 1;

  END WHILE;

  -- Berechne matrix_inverse. Verwende dazu den in der Arbeit referenzierten Algorithmus.
  INSERT INTO matrix_inverse
  SELECT *
  FROM matrix_product_1;

  SET counter_1 = 0;

  WHILE counter_1 < n DO

    SET counter_1 = counter_1 + 1;

    DROP TEMPORARY TABLE IF EXISTS pivot_row;
    CREATE TEMPORARY TABLE pivot_row (
        `column` INT(11),
        `value` DECIMAL(65, 30)
    );

    INSERT INTO pivot_row
    SELECT `column`, `value`
    FROM matrix_inverse
    WHERE `row` = counter_1;

    SET pivot = (
        SELECT `value`
        FROM matrix_inverse
        WHERE `row` = counter_1 AND `column` = counter_1
    );

    UPDATE matrix_inverse
    SET `value` = `value` / pivot
    WHERE `row` = counter_1 AND `column` <> counter_1;

    UPDATE matrix_inverse
    SET `value` = - `value` / pivot
    WHERE `row` <> counter_1 AND `column` = counter_1;

    SET counter_2 = 1;

    WHILE counter_2 <= n DO

      IF counter_2 <> counter_1 THEN

        SET counter_3 = 1;

        WHILE counter_3 <= n DO

          IF counter_3 <> counter_1 THEN

            SET pivot = (
              SELECT `value`
              FROM pivot_row
              WHERE `column` = counter_3
            ) * (
              SELECT `value`
              FROM matrix_inverse
              WHERE `row` = counter_2 AND `column` = counter_1
            );

            UPDATE matrix_inverse
            SET `value` = `value` + pivot
            WHERE `row` = counter_2 AND `column` = counter_3;

          END IF;

          SET counter_3 = counter_3 + 1;

        END WHILE;

      END IF;

      SET counter_2 = counter_2 + 1;

    END WHILE;

    UPDATE matrix_inverse
    SET `value` = 1 / `value`
    WHERE `row` = counter_1 AND `column` = counter_1;

  END WHILE;

  -- Berechne matrix_product_2. Iteriere dazu über alle Zeilen der Ergebnismatrix.
  SET counter_1 = 1;

  WHILE counter_1 <= n DO

    -- Berechne den Wert des aktuellen Matrixelements.
    INSERT INTO matrix_product_2 VALUES (
      counter_1,
      1,
      (
        SELECT SUM(matrix_y.`value` * matrix_transposed.`value`)
        FROM matrix_y, matrix_transposed
        WHERE matrix_transposed.`row` = counter_1
          AND matrix_transposed.`column` = matrix_y.`row`
      )
    );

    SET counter_1 = counter_1 + 1;

  END WHILE;

  -- Berechne matrix_result. Iteriere dazu über alle Zeilen der Ergebnismatrix.
  SET counter_1 = 1;

  WHILE counter_1 <= n DO

    -- Berechne den Wert des aktuellen Matrixelements.
    INSERT INTO matrix_result VALUES (
      counter_1,
      1,
      (
        SELECT SUM(matrix_product_2.`value` * matrix_inverse.`value`)
        FROM matrix_product_2, matrix_inverse
        WHERE matrix_inverse.`row` = counter_1
          AND matrix_inverse.`column` = matrix_product_2.`row`
      )
    );

    SET counter_1 = counter_1 + 1;

  END WHILE;

  -- Gib eine Relation mit Parameternamen und zugehörigen Werten zurück.
  SELECT
    CASE row
      WHEN 1 THEN 'alpha'
      WHEN 2 THEN 'beta_purchases'
      WHEN 3 THEN 'beta_age'
    END AS `variable`,
    value
  FROM matrix_result;

  -- Lösche die temporären Relationen wieder.
  DROP TEMPORARY TABLE IF EXISTS matrix_X;
  DROP TEMPORARY TABLE IF EXISTS matrix_transposed;
  DROP TEMPORARY TABLE IF EXISTS matrix_product_1;
  DROP TEMPORARY TABLE IF EXISTS matrix_inverse;
  DROP TEMPORARY TABLE IF EXISTS matrix_product_2;
  DROP TEMPORARY TABLE IF EXISTS matrix_y;
  DROP TEMPORARY TABLE IF EXISTS matrix_result;

  END;;

  DELIMITER ;
\end{minted}

\section{Logistische Regression}
\label{appendix:D:3}

\begin{minted}[linenos,breaklines,frame=single]{sql}
  -- Lösche die bestehenden Prozeduren, falls vorhanden.
  DROP PROCEDURE IF EXISTS calculate_gradient;
  DROP PROCEDURE IF EXISTS calculate_new_parameters;
  DROP PROCEDURE IF EXISTS calculate_logit;
  DROP PROCEDURE IF EXISTS logistic_regression;

  DELIMITER ;;
  -- Erstelle eine Prozedur zur Berechnung der Werte der logistischen Funktion.
  CREATE PROCEDURE `calculate_logit`()
  BEGIN

  -- Deklariere die benötigten Variablen.
  DECLARE alpha_old DECIMAL(65, 30);
  DECLARE alpha_new DECIMAL(65, 30);

  -- Bestimme den alten und neuen Wert von alpha.
  SET alpha_old = (
    SELECT old
    FROM parameters
    WHERE variable = 'alpha'
  );
  SET alpha_new = (
    SELECT new
    FROM parameters
    WHERE variable = 'alpha'
  );

  -- Berechne die Werte der logistischen Funktion für alle Datenpunkte.
  DELETE FROM logits;
  INSERT INTO logits
    SELECT
      d.id,
      1 / (1 + exp(- SUM(d.value * p.old) - alpha_old)) AS `old`,
      1 / (1 + exp(- SUM(d.value * p.new) - alpha_new)) AS `new`
    FROM datapoints d
    JOIN parameters p ON p.variable = d.variable
    GROUP BY d.id;

  END;;

  -- Erstelle eine Prozedur zur Berechnung des Gradienten.
  CREATE PROCEDURE `calculate_gradient`()
  BEGIN

  DELETE FROM gradient;

  -- Berechne die partielle Ableitung nach alpha.
  INSERT INTO gradient
  SELECT 'alpha' AS `variable`, SUM(bv.value - l.old) AS `value`
  FROM logits l
  JOIN binary_values bv ON bv.id = l.id;

  -- Berechne die partielle Ableitung nach allen beta-Parametern.
  INSERT INTO gradient
  SELECT d.variable, SUM(d.value * (bv.value - l.old)) AS `value`
  FROM logits l
  JOIN binary_values bv ON bv.id = l.id
  JOIN datapoints d ON d.id = l.id
  GROUP BY d.variable;

  END;;

  -- Erzeuge eine Prozedur zur Berechnung der neuen Parameter abhängig von der aktuellen Schrittweite.
  CREATE PROCEDURE `calculate_new_parameters`(IN step DECIMAL(65, 30))
  BEGIN

  UPDATE parameters
  JOIN gradient ON gradient.variable = parameters.variable
  SET parameters.new = parameters.old + step * gradient.value;

  END;;

  -- Erstelle die Prozedur für logistische Regression.
  CREATE PROCEDURE `logistic_regression`(IN number_datapoints INT(11), IN rounds INT(11), step DECIMAL(65, 30))
  BEGIN

  -- Deklariere die verwendeten Variablen.
  DECLARE min INT(11);
  DECLARE max INT(11);
  DECLARE transform DECIMAL(65, 30);
  DECLARE better INT(1);
  DECLARE counter INT(11);

  -- Erstelle eine temporäre Relation für die Werte der unabhängigen Variablen.
  DROP TEMPORARY TABLE IF EXISTS datapoints;
  CREATE TEMPORARY TABLE datapoints (
    id INT(11),
    variable VARCHAR(32),
    value DECIMAL(65, 30),
    PRIMARY KEY (id, variable)
  );

  -- Berechne Minimum und Maximum der unabhängigen Variable.
  SET min = (SELECT MIN(money) FROM sample);
  SET max = (SELECT MAX(money) FROM sample);

  -- Füge die linear transformierten Werte der unabhängigen Variablen in die Relation datapoints ein.
  SET @counter = 0;
  INSERT INTO datapoints
  SELECT
    @counter := @counter + 1 AS `id`,
    'beta_money' AS `variable`,
    (money - min) / (max - min) AS `value`
  FROM sample
  LIMIT number_datapoints;

  -- Erstelle eine temporäre Relation für die (binären) Werte der abhängigen Variablen.
  DROP TEMPORARY TABLE IF EXISTS binary_values;
  CREATE TEMPORARY TABLE binary_values (
    id INT(11),
    value INT(1),
    PRIMARY KEY (id)
  );

  -- Füge die Werte der abhängingen Variable ein.
  SET @counter = 0;
  INSERT INTO binary_values
  SELECT
    @counter := @counter + 1 AS `id`,
    premium AS `value`
  FROM sample
  LIMIT number_datapoints;

  -- Erstelle eine temporäre Relation für die alten und neuen Parameterwerte.
  DROP TEMPORARY TABLE IF EXISTS parameters;
  CREATE TEMPORARY TABLE parameters (
    variable VARCHAR(32),
    old DECIMAL(65, 30),
    new DECIMAL(65, 30),
    PRIMARY KEY (variable)
  );

  -- Füge die Initialwerte der Parameter ein.
  INSERT INTO parameters VALUES
    ('alpha', 0, 0),
    ('beta_money', 0, 0);

  -- Erstelle eine temporäre Relation für die Werte der logistischen Funktion für alle Datenpunkte.
  DROP TEMPORARY TABLE IF EXISTS logits;
  CREATE TEMPORARY TABLE logits (
    id INT(11),
    old DECIMAL(65, 30),
    new DECIMAL(65, 30),
    PRIMARY KEY (id)
  );

  -- Befülle die Relation für die Werte der logistischen Funktion.
  CALL calculate_logit();

  -- Erstelle eine teporäre Relation für den Gradienten.
  DROP TEMPORARY TABLE IF EXISTS gradient;
  CREATE TEMPORARY TABLE gradient (
    variable VARCHAR(32),
    value DECIMAL(65, 30),
    PRIMARY KEY (variable)
  );

  -- Iteriere über die Anzahl der gewünschten Iterationen.
  SET counter = 0;
  WHILE counter < rounds AND step > 0.000000000000000000000000000001 DO

    -- Berechne den Gradienten und die neuen Parameter mit der aktuellen Schrittweite.
    CALL calculate_gradient();
    CALL calculate_new_parameters(step);
    CALL calculate_logit();

    -- Verringere die Schrittweite solange, bis die neuen Parameter ein besseres Ergebnis liefern als die alten.
    WHILE (
      SELECT
        SUM(LOG(bv.value * l.new + (1 - bv.value) * (1 - l.new))) >
        SUM(LOG(bv.value * l.old + (1 - bv.value) * (1 - l.old)))
      FROM logits l
      JOIN binary_values bv ON bv.id = l.id
    ) = 0 AND step > 0.000000000000000000000000000001 DO

      SET step = step / 2;
      CALL calculate_new_parameters(step);
      CALL calculate_logit();

    END WHILE;

    -- Ersetze die alten Werte durch die neuen Werte.
    UPDATE parameters
    SET parameters.old = parameters.new;

    UPDATE logits
    SET logits.old = logits.new;

    SET counter = counter + 1;

  END WHILE;

  -- Transformiere die Parameter linear, um den originalen Daten zu entsprechen.
  UPDATE parameters
  SET old = old / (max - min)
  WHERE variable = 'beta_money';

  SET transform = (SELECT old FROM parameters WHERE variable = 'beta_money');

  UPDATE parameters
  SET old = old - transform * min
  WHERE variable = 'alpha';

  -- Gib eine Relation mit Parametername und zugehörigem Wert zurück.
  SELECT variable, old AS `value`
  FROM parameters;

  -- Lösche die temporären Relationen wieder.
  DROP TEMPORARY TABLE IF EXISTS datapoints;
  DROP TEMPORARY TABLE IF EXISTS binary_values;
  DROP TEMPORARY TABLE IF EXISTS parameters;
  DROP TEMPORARY TABLE IF EXISTS logits;
  DROP TEMPORARY TABLE IF EXISTS gradient;

  END;;
  DELIMITER ;
\end{minted}

\chapter{PostgreSQL-Skripte}
\label{appendix:E}

\section{Einfache lineare Regression}
\label{appendix:E:1}

\begin{minted}[linenos,breaklines,frame=single]{sql}
  -- Erstelle (oder ersetze falls vorhanden) die Prozedur für einfache lineare Regression.
  CREATE OR REPLACE FUNCTION simple_linear_regression(number_datapoints INTEGER)
  RETURNS TABLE (
    variable VARCHAR(50),
    value NUMERIC(65, 30)
  ) AS $$
  BEGIN

  -- Erstelle eine Relation für die zu verwendenden Datenpunkte.
  DROP TABLE IF EXISTS datapoints;
  CREATE TEMPORARY TABLE datapoints (
    purchases INTEGER,
    money INTEGER
  );

  -- Füge die gewünschte Anzahl der Datenpunkte in die temporäre Relation ein.
  INSERT INTO datapoints
  SELECT purchases, money
  FROM sample
  LIMIT number_datapoints;

  RETURN QUERY
  WITH
    -- Berechne die Mittelwerte der abhängigen und unabhängigen Variable.
    means AS (
      SELECT
        AVG(purchases) AS mean_purchases,
        AVG(money) AS mean_money
      FROM datapoints
    ),
    -- Berechne die Summen im Nenner und Zähler der Formel für beta.
    sums AS (
      SELECT
        SUM((purchases - mean_purchases) * (money - mean_money)) AS nominator,
        SUM(POWER(purchases - mean_purchases, 2)) AS denominator
      FROM datapoints, means
    ),
    -- Berechne beta.
    beta AS (
      SELECT
        'beta'::VARCHAR(50) AS variable,
        nominator / denominator AS value
      FROM sums
    ),
    -- Berechne alpha.
    alpha AS (
      SELECT
        'alpha'::VARCHAR(50) AS variable,
        mean_money - beta.value * mean_purchases AS value
      FROM means, beta
    )
  -- Gib eine Relation mit Parametername und zugehörigem Wert zurück.
  SELECT *
  FROM alpha
  UNION
  SELECT *
  FROM beta;

  -- Lösche die temporäre Relation mit den Datenpunkten wieder.
  DROP TABLE IF EXISTS datapoints;

  END;
  $$ LANGUAGE plpgsql;
\end{minted}

\section{Multiple lineare Regression}
\label{appendix:E:2}

\begin{minted}[linenos,breaklines,frame=single]{sql}
-- Erstelle Funktion für die Berechnung der transponierten Matrix.
CREATE OR REPLACE FUNCTION matrix_transpose(a NUMERIC(65, 30)[][])
RETURNS NUMERIC(65, 30)[][] AS $$
DECLARE
  rows_a INTEGER := array_length(a, 1);
  columns_a INTEGER := array_length(a, 2);
  i INTEGER;
  j INTEGER;
  c NUMERIC(65, 30)[][];
  new_row NUMERIC(65, 30)[];
BEGIN

-- Iteriere über alle Zeilen und Spalten der ursprüglichen Matrix.
i := 1;
WHILE i <= columns_a LOOP

  j := 1;
  WHILE j <= rows_a LOOP
    -- Erzeuge ein Array mit der neuen Zeile der transponierten Matrix aus der Spalte der ursprünglichen Matrix.
    new_row[j] := a[j][i];
    j := j + 1;
  END LOOP;

  -- Füge die Zeile in die Ergebnismatrix ein.
  c := array_cat(c, array[new_row]);
  i := i + 1;
END LOOP;

RETURN c;

END;
$$ LANGUAGE plpgsql;

-- Erstelle Funktion für die Berechnung des Produktes zweier Matrizen.
CREATE OR REPLACE FUNCTION matrix_multiplication(a NUMERIC(65, 30)[][], b NUMERIC(65, 30)[][])
RETURNS NUMERIC(65, 30)[][] AS $$
DECLARE
  rows_a INTEGER := array_length(a, 1);
  columns_a INTEGER := array_length(a, 2);
  columns_b INTEGER := array_length(b, 2);
  new_row NUMERIC(65, 30)[];
  c NUMERIC(65, 30)[][];
  counter_1 INTEGER;
  counter_2 INTEGER;
  counter_3 INTEGER;
BEGIN

-- Iteriere über die Zeilen und Spalten der Ergebnismatrix.
counter_1 := 1;
WHILE counter_1 <= rows_a LOOP

  counter_2 := 1;
  WHILE counter_2 <= columns_b LOOP

    -- Initiiere den Wert des aktuellen Elementes der Ergebnismatrix mit 0.
    new_row[counter_2] := 0;

    -- Iteriere über die Summanden zur Berechnung des aktuellen Matrixelements.
    counter_3 := 1;
    WHILE counter_3 <= columns_a LOOP
      -- Addiere den aktuellen Summanden zum Wert des aktuellen Elementes.
      new_row[counter_2] := new_row[counter_2] + a[counter_1][counter_3] * b[counter_3][counter_2];
      counter_3 := counter_3 + 1;
    END LOOP;

    counter_2 := counter_2 + 1;
  END LOOP;

  c := array_cat(c, array[new_row]);
  counter_1 := counter_1 + 1;
END LOOP;

RETURN c;

END;
$$ LANGUAGE plpgsql;

-- Erstelle Funktion für die Berechnung der inversen Matrix.
CREATE OR REPLACE FUNCTION matrix_inversion(a NUMERIC(65, 30)[][])
RETURNS NUMERIC(65, 30)[][] AS $$
DECLARE
  n INTEGER := array_length(a, 1);
  p INTEGER := 0;
  i INTEGER;
  j INTEGER;
  c NUMERIC(65, 30)[][] := a;
  o NUMERIC(65, 30)[][];
BEGIN

-- Verwende den in der Arbeit referenzierten Algorithmus.
WHILE p < n LOOP

  p := p + 1;

  o := c;

  j := 1;
  WHILE j <= n LOOP
    IF j <> p THEN
      c[p][j] := c[p][j] / c[p][p];
    END IF;
    j := j + 1;
  END LOOP;

  i := 1;
  WHILE i <= n LOOP
    IF i <> p THEN
      c[i][p] := - c[i][p] / c[p][p];
    END IF;
    i := i + 1;
  END LOOP;

  i := 1;
  WHILE i <= n LOOP
    IF i <> p THEN
      j := 1;
      WHILE j <= n LOOP
        IF j <> p THEN
          c[i][j] := c[i][j] + o[p][j] * c[i][p];
        END IF;
        j := j + 1;
      END LOOP;
    END IF;
    i := i + 1;
  END LOOP;

  c[p][p] := 1 / c[p][p];

END LOOP;

RETURN c;

END;
$$ LANGUAGE plpgsql;

-- Erstelle die Funktion für multiple lineare Regression.
CREATE OR REPLACE FUNCTION multiple_linear_regression(number_datapoints INTEGER)
RETURNS TABLE (
  variable VARCHAR(50),
  value NUMERIC(65, 30)
) AS $$
DECLARE
  -- Erzeuge die Matrix X mit der gewünschten Anzahl an Datenpunkten.
  x INTEGER[][]:= (
    SELECT ARRAY(
      SELECT ARRAY[1, purchases, age]
      FROM sample
      LIMIT number_datapoints
    )
  );
  -- Erzeuge die Matrix y mit der gewünschten Anzahl an Datenpunkten.
  y INTEGER[]:= (
    SELECT ARRAY(
      SELECT ARRAY[money]
      FROM sample
      LIMIT number_datapoints
    )
  );
  b NUMERIC(65, 30)[][];
BEGIN

-- Berechne die Lösungsformel unter Verwendung der zuvor definierten Funktionen.
b := matrix_multiplication(
  matrix_inversion(
    matrix_multiplication(
      matrix_transpose(x),
      x
    )
  ),
  matrix_multiplication(
    matrix_transpose(x),
    y
  )
);

-- Gib eine Relation mit Parameternamen und zugehörigen Werten zurück.
RETURN QUERY
SELECT 'alpha'::VARCHAR(50) AS variable, b[1][1] AS value
UNION
SELECT 'beta_purchases'::VARCHAR(50) AS variable, b[2][1] AS value
UNION
SELECT 'beta_age'::VARCHAR(50) AS variable, b[3][1] AS value;

END;
$$ LANGUAGE plpgsql;
\end{minted}

\section{Logistische Regression}
\label{appendix:E:3}

\begin{minted}[linenos,breaklines,frame=single]{sql}
  -- Erstelle (oder ersetzte falls vorhanden) eine Prozedur zur Berechnung der Werte der logistischen Funktion.
  CREATE OR REPLACE FUNCTION calculate_logit()
  RETURNS void AS $$
  BEGIN

  DELETE FROM logits;

  WITH
    -- Bestimme den alten und neuen Wert von alpha.
    alpha_old AS (
      SELECT old
      FROM parameters
      WHERE variable = 'alpha'
    ),
    alpha_new AS (
      SELECT new
      FROM parameters
      WHERE variable = 'alpha'
    )
  -- Berechne die Werte der logistischen Funktion für alle Datenpunkte.
  INSERT INTO logits
    SELECT
      d.id,
      1 / (1 + EXP(- SUM(d.value * p.old) - (SELECT old FROM alpha_old))) AS old,
      1 / (1 + EXP(- SUM(d.value * p.new) - (SELECT new FROM alpha_new))) AS new
    FROM datapoints d
    JOIN parameters p ON p.variable = d.variable
    GROUP BY d.id;

  RETURN;

  END;
  $$ LANGUAGE plpgsql;

  -- Erstelle (oder ersetze falls vorhanden) eine Prozedur zur Berechnung des Gradienten.
  CREATE OR REPLACE FUNCTION calculate_gradient()
  RETURNS void AS $$
  BEGIN

  DELETE FROM gradient;

  -- Berechne die partielle Ableitung nach alpha.
  INSERT INTO gradient
  SELECT 'alpha' AS variable, SUM(bv.value - l.old) AS value
  FROM logits l
  JOIN binary_values bv ON bv.id = l.id;

  -- Berechne die partielle Ableitung nach allen beta-Parametern.
  INSERT INTO gradient
  SELECT d.variable, SUM(d.value * (bv.value - l.old)) AS value
  FROM logits l
  JOIN binary_values bv ON bv.id = l.id
  JOIN datapoints d ON d.id = l.id
  GROUP BY d.variable;

  RETURN;

  END;
  $$ LANGUAGE plpgsql;

  -- Erzeuge (oder ersetze falls vorhanden) eine Prozedur zur Berechnung der neuen Parameter abhängig von der aktuellen Schrittweite.
  CREATE OR REPLACE FUNCTION calculate_new_parameters(step NUMERIC(65, 30))
  RETURNS void AS $$
  BEGIN

  UPDATE parameters
  SET new = old + step * gradient.value
  FROM gradient
  WHERE gradient.variable = parameters.variable;

  RETURN;

  END;
  $$ LANGUAGE plpgsql;

  -- Erstelle die Prozedur für logistische Regression.
  CREATE OR REPLACE FUNCTION logistic_regression(number_datapoints INTEGER, rounds INTEGER, step NUMERIC(65, 30))
  RETURNS TABLE (
    variable VARCHAR(50),
    value NUMERIC(65, 30)
  ) AS $$
  DECLARE
    counter INTEGER;
  BEGIN

  -- Erstelle eine Relation für die Werte der unabhängigen Variablen.
  DROP TABLE IF EXISTS datapoints;
  CREATE TEMPORARY TABLE datapoints (
    id INTEGER,
    variable VARCHAR(50),
    value NUMERIC(65, 30)
  );

  -- Füge die linear transformierten Werte der unabhängigen Variablen in die Relation datapoints ein.
  INSERT INTO datapoints
  SELECT
    row_number() OVER () AS id,
    'beta_money' AS variable,
    (money - (
      SELECT MIN(money) FROM sample
    ))::NUMERIC(65, 30) / ((
      SELECT MAX(money) FROM sample
    ) - (
      SELECT MIN(money) FROM sample
    ))::NUMERIC(65, 30) AS value
  FROM sample
  LIMIT number_datapoints;

  -- Erstelle eine Relation für die (binären) Werte der abhängigen Variablen.
  DROP TABLE IF EXISTS binary_values;
  CREATE TEMPORARY TABLE binary_values (
    id INTEGER,
    value INTEGER
  );

  -- Füge die Werte der abhängingen Variable ein.
  INSERT INTO binary_values
  SELECT
    row_number() OVER () AS id,
    premium AS value
  FROM sample
  LIMIT number_datapoints;

  -- Erstelle eine Relation für die alten und neuen Parameterwerte.
  DROP TABLE IF EXISTS parameters;
  CREATE TEMPORARY TABLE parameters (
    variable VARCHAR(50),
    old NUMERIC(65, 30),
    new NUMERIC(65, 30)
  );

  -- Füge die Initialwerte der Parameter ein.
  INSERT INTO parameters VALUES
    ('alpha', 0, 0),
    ('beta_money', 0, 0);

  -- Erstelle eine Relation für die Werte der logistischen Funktion für alle Datenpunkte.
  DROP TABLE IF EXISTS logits;
  CREATE TEMPORARY TABLE logits (
    id INTEGER,
    old NUMERIC(65, 30),
    new NUMERIC(65, 30)
  );

  -- Befülle die Relation für die Werte der logistischen Funktion.
  PERFORM calculate_logit();

  -- Erstelle eine Relation für den Gradienten.
  DROP TABLE IF EXISTS gradient;
  CREATE TEMPORARY TABLE gradient (
    variable VARCHAR(50),
    value NUMERIC(65, 30)
  );

  -- Iteriere über die Anzahl der gewünschten Iterationen.
  counter := 0;
  WHILE counter < rounds AND step > 0.000000000000000000000000000001 LOOP

    -- Berechne den Gradienten und die neuen Parameter mit der aktuellen Schrittweite.
    PERFORM calculate_gradient();
    PERFORM calculate_new_parameters(step);
    PERFORM calculate_logit();

    -- Verringere die Schrittweite solange, bis die neuen Parameter ein besseres Ergebnis liefern als die alten.
    WHILE NOT (
      SELECT
        SUM(LOG(bv.value * l.new + (1 - bv.value) * (1 - l.new))) >
        SUM(LOG(bv.value * l.old + (1 - bv.value) * (1 - l.old)))
      FROM logits l
      JOIN binary_values bv ON bv.id = l.id
    ) AND step > 0.000000000000000000000000000001 LOOP

      step := step / 2;
      PERFORM calculate_new_parameters(step);
      PERFORM calculate_logit();

    END LOOP;

    -- Ersetze die alten Werte durch die neuen Werte.
    UPDATE parameters
    SET old = new;

    UPDATE logits
    SET old = new;

    counter := counter + 1;

  END LOOP;

  -- Transformiere die Parameter linear, um den originalen Daten zu entsprechen.
  UPDATE parameters
  SET old = old / ((SELECT MAX(money) FROM sample) - (SELECT MIN(money) FROM sample))
  WHERE parameters.variable = 'beta_money';

  UPDATE parameters
  SET old = old - (SELECT old FROM parameters p2 WHERE p2.variable = 'beta_money') * (SELECT MIN(money) FROM sample)
  WHERE parameters.variable = 'alpha';

  -- Gib eine Relation mit Parametername und zugehörigem Wert zurück.
  RETURN QUERY
  SELECT parameters.variable::VARCHAR(50), old AS value
  FROM parameters;

  -- Lösche die Relationen wieder.
  DROP TABLE IF EXISTS datapoints;
  DROP TABLE IF EXISTS binary_values;
  DROP TABLE IF EXISTS parameters;
  DROP TABLE IF EXISTS logits;
  DROP TABLE IF EXISTS gradient;

  END;
  $$ LANGUAGE plpgsql;
\end{minted}

\chapter{Python-Skripte für das Benchmarking}
\label{appendix:F}

\section{Berechnung der Benchmarks}
\label{appendix:F:1}

\begin{minted}[linenos,breaklines,frame=single]{python}
  from time import time
  from subprocess import call
  import os
  import sys
  import csv
  import json

  # Erzeuge Funktion, um einen Fortschrittsbalken zu drucken.
  def print_progessbar(iterations, progress):
    sys.stdout.write("\033[100D")
    for i in range(progress * 50 // iterations): sys.stdout.write("#")
    for i in range(50 - (progress * 50 // iterations)): sys.stdout.write(".")
    sys.stdout.write("  ||  %i%% / 100%%" % (progress * 100 // iterations))
    sys.stdout.flush()

  # Erzeuge Funktion, die eine bestimme Anzahl an Laufzeiten für eine bestimmte Anzahl an Datenpunkten für eine bestimme Art der Regression in einer bestimmten Sprache berechnet.
  def benchmark(type_regression, language, command, set_number_datapoints, file, iterations):
    print("Evaluating %s in %s..." % (type_regression, language))

    # Iteriere über ein übergenenes Array mit den Anzahlen der zu verwendenden Datenpunkte.
    for number_datapoints in set_number_datapoints:
      print("Use %s datapoints:" % (number_datapoints))

      # Erzeuge eine Datei, in die die Ausgaben aus stdout und stderr geschrieben werden.
      logfile = open("logs/%s-%s-%i.txt" % (language, type_regression.replace(" ", "-"), number_datapoints), "a")

      # Füge die Anzahl der Datenpunkte in den Kommandozeilen-Befehl ein.
      if command.count("%") == 2:
        exec_command = command % (number_datapoints, 8 / number_datapoints)
      else:
        exec_command = command % number_datapoints

      # Iteriere über die Anzahl der gewünschten Iterationen.
      for i in range(iterations):
        # Drucke den Fortschrittsbalken
        print_progessbar(iterations, i)

        # Speichere die Startzeit.
        start_time = time()

        # Führe den Kommandozeilen-Befehl aus.
        call(exec_command, stdout = logfile, stderr = logfile, shell = True)

        # Speichere die Endzeit.
        end_time = time()

        # Füge eine neue Zeile in der Benchmark-Datei ein.
        file.write("%s,%s,%i,%s\n" % (
          language,
          type_regression,
          number_datapoints,
          (end_time - start_time)
        ))

      # Schließe die Log-Datei.
      logfile.close()

      # Drucke 100% im Forschrittbalken und eine leere Zeile.
      print_progessbar(iterations, iterations)
      print("")
    return

  def main(argv):
    # Bestimme default-Werte.
    iterations = 100
    set_number_datapoints = [10, 100, 1000, 10000, 100000]
    run_r = False
    run_tensorflow = False
    run_mysql = False
    run_postgresql = False
    run_simple_linear_regression = False
    run_multiple_linear_regression = False
    run_logistic_regression = False

    # Durchlaufe die übergebenen Argumente.
    for arg in argv:
      # Fixiere die Anzahl der Datenpunkte.
      if "--datapoints=" in arg: set_number_datapoints = [int(arg.split("=")[1])]

      # Setzte die Anzahl der Iterationen.
      if "--iterations=" in arg: iterations = int(arg.split("=")[1])

      # Berechne Benchmarks für R.
      if arg in ["--r", "-r"]: run_r = True

      # Verwende TensorFlow.
      if arg in ["--tensorflow", "-t"]: run_tensorflow = True

      # Berechne Benchmarks für MySQL.
      if arg in ["--mysql", "-m"]: run_mysql = True

      # Berechne Benchmarks für PostgreSQL.
      if arg in ["--postgresql", "-p"]: run_postgresql = True

      # Berechne Benchmarks für einfache lineare Regression.
      if arg in ["--simple-linear", "-slr"]: run_simple_linear_regression = True

      # Berechne Benchmarks für multiple lineare Regression.
      if arg in ["--multiple-linear", "-mlr"]: run_multiple_linear_regression = True

      # Berechne Benchmarks für logistische Regression.
      if arg in ["--logistic", "-lr"]: run_logistic_regression = True

    # Wenn keine Art der Regression und keine Sprache spezifiziert wurde, berechne alles.
    if not ((
      run_r or
      run_tensorflow or
      run_mysql or
      run_postgresql
    ) and (
      run_simple_linear_regression or
      run_multiple_linear_regression or
      run_logistic_regression
    )):
      run_r = True
      run_tensorflow = True
      run_mysql = True
      run_postgresql = True
      run_simple_linear_regression = True
      run_multiple_linear_regression = True
      run_logistic_regression = True

    # Erzeuge eine Datei für die berechneten Laufzeiten und schreibe die erste Zeile.
    file = open("benchmarks-%i.csv" % (time()), "w")
    file.write("language,type,datapoints,time\n")

    # Berechne einfache lineare Regression in R.
    if run_r and run_simple_linear_regression:
      benchmark(
        "simple linear regression",
        "r",
        "Rscript r/simpleLinearRegression.R %i -",
        set_number_datapoints,
        file,
        iterations
      )

    # Berechne multiple lineare Regression in R.
    if run_r and run_multiple_linear_regression:
      benchmark(
        "multiple linear regression",
        "r",
        "Rscript r/multipleLinearRegression.R %i -",
        set_number_datapoints,
        file,
        iterations
      )

    # Berechne logistische Regression in R.
    if run_r and run_logistic_regression:
      benchmark(
        "logistic regression",
        "r",
        "Rscript r/logisticRegression.R %i -",
        set_number_datapoints,
        file,
        iterations
      )

    # Berechne einfache lineare Regression in TensorFlow.
    if run_tensorflow and run_simple_linear_regression:
      benchmark(
        "simple linear regression",
        "tensorflow",
        "python3 tensorflow/simpleLinearRegression.py %i -",
        set_number_datapoints,
        file,
        iterations
      )

    # Berechne multiple lineare Regression in TensorFlow.
    if run_tensorflow and run_multiple_linear_regression:
      benchmark(
        "multiple linear regression",
        "tensorflow",
        "python3 tensorflow/multipleLinearRegression.py %i -",
        set_number_datapoints,
        file,
        iterations
      )

    # Berechne logistische Regression in TensorFlow.
    if run_tensorflow and run_logistic_regression:
      benchmark(
        "logistic regression",
        "tensorflow",
        "python3 tensorflow/logisticRegression.py %i -",
        set_number_datapoints,
        file,
        iterations
      )

    # Berechne einfache lineare Regression in MySQL.
    if run_mysql and run_simple_linear_regression:
      benchmark(
        "simple linear regression",
        "mysql",
        "echo \"CALL regression.simple_linear_regression(%i)\" | " + "mysql -u %s -p%s" % (
          json.loads(os.environ["MYSQL_CONFIG"])["user"],
          json.loads(os.environ["MYSQL_CONFIG"])["password"]
        ),
        set_number_datapoints,
        file,
        iterations
      )

    # Berechne multiple lineare Regression in MySQL.
    if run_mysql and run_multiple_linear_regression:
      benchmark(
        "multiple linear regression",
        "mysql",
        "echo \"CALL regression.multiple_linear_regression(%i)\" | " + "mysql -u %s -p%s" % (
          json.loads(os.environ["MYSQL_CONFIG"])["user"],
          json.loads(os.environ["MYSQL_CONFIG"])["password"]
        ),
        set_number_datapoints,
        file,
        iterations
      )

    # Berechne logistische Regression in MySQL.
    if run_mysql and run_logistic_regression:
      benchmark(
        "logistic regression",
        "mysql",
        "echo \"CALL regression.logistic_regression(%i, 1000, %f)\" | " + "mysql -u %s -p%s" % (
          json.loads(os.environ["MYSQL_CONFIG"])["user"],
          json.loads(os.environ["MYSQL_CONFIG"])["password"]
        ),
        set_number_datapoints,
        file,
        iterations
      )

    # Berechne einfache lineare Regression in PostgreSQL.
    if run_postgresql and run_simple_linear_regression:
      benchmark(
        "simple linear regression",
        "postgresql",
        "echo \"SELECT simple_linear_regression(%i)\" | psql regression",
        set_number_datapoints,
        file,
        iterations
      )

    # Berechne multiple lineare Regression in PostgreSQL.
    if run_postgresql and run_multiple_linear_regression:
      benchmark(
        "multiple linear regression",
        "postgresql",
        "echo \"SELECT multiple_linear_regression(%i)\" | psql regression",
        set_number_datapoints,
        file,
        iterations
      )

    # Berechne logistische Regression in PostgreSQL.
    if run_postgresql and run_logistic_regression:
      benchmark(
        "logistic regression",
        "postgresql",
        "echo \"SELECT logistic_regression(%i, 1000, %f)\" | psql regression",
        set_number_datapoints,
        file,
        iterations
      )

    # Schließe die Benchmark-Datei.
    file.close()
    return

  # Führe die main-Funktion aus.
  if __name__ == "__main__":
    main(sys.argv)
\end{minted}

\section{Auswertung der Benchmarks}
\label{appendix:F:2}

\begin{minted}[linenos,breaklines,frame=single]{python}
  import sys
  import os.path as p
  import csv
  import matplotlib.pyplot as plt

  # Erzeuge Funktion, die die Benchmarks aus der csv-Datei einliest un gruppiert.
  def calculate_benchmarks():
    # Bestimme den Dateipfad der benchmark-Datei und öffne diese.
    filename = p.abspath(p.join(p.dirname(p.realpath(__file__)), "benchmarks.csv"))
    csvfile = open(filename, newline="")
    csvreader = csv.reader(csvfile, delimiter=",", quotechar="|")

    # Erstelle ein dictionary, in dem die Laufzeiten aggregiert werden sollen.
    count = {
      "simple linear regression": {
        "10": {
          "r": {"count": 0, "time": 0},
          "tensorflow": {"count": 0, "time": 0},
          "mysql": {"count": 0, "time": 0},
          "postgresql": {"count": 0, "time": 0}
        },
        "100": {
          "r": {"count": 0, "time": 0},
          "tensorflow": {"count": 0, "time": 0},
          "mysql": {"count": 0, "time": 0},
          "postgresql": {"count": 0, "time": 0}
        },
        "1000": {
          "r": {"count": 0, "time": 0},
          "tensorflow": {"count": 0, "time": 0},
          "mysql": {"count": 0, "time": 0},
          "postgresql": {"count": 0, "time": 0}
        },
        "10000": {
          "r": {"count": 0, "time": 0},
          "tensorflow": {"count": 0, "time": 0},
          "mysql": {"count": 0, "time": 0},
          "postgresql": {"count": 0, "time": 0}
        },
        "100000": {
          "r": {"count": 0, "time": 0},
          "tensorflow": {"count": 0, "time": 0},
          "mysql": {"count": 0, "time": 0},
          "postgresql": {"count": 0, "time": 0}
        }
      },
      "multiple linear regression": {
        "10": {
          "r": {"count": 0, "time": 0},
          "tensorflow": {"count": 0, "time": 0},
          "mysql": {"count": 0, "time": 0},
          "postgresql": {"count": 0, "time": 0}
        },
        "100": {
          "r": {"count": 0, "time": 0},
          "tensorflow": {"count": 0, "time": 0},
          "mysql": {"count": 0, "time": 0},
          "postgresql": {"count": 0, "time": 0}
        },
        "1000": {
          "r": {"count": 0, "time": 0},
          "tensorflow": {"count": 0, "time": 0},
          "mysql": {"count": 0, "time": 0},
          "postgresql": {"count": 0, "time": 0}
        },
        "10000": {
          "r": {"count": 0, "time": 0},
          "tensorflow": {"count": 0, "time": 0},
          "mysql": {"count": 0, "time": 0},
          "postgresql": {"count": 0, "time": 0}
        },
        "100000": {
          "r": {"count": 0, "time": 0},
          "tensorflow": {"count": 0, "time": 0},
          "mysql": {"count": 0, "time": 0},
          "postgresql": {"count": 0, "time": 0}
        }
      },
      "logistic regression": {
        "10": {
          "r": {"count": 0, "time": 0},
          "tensorflow": {"count": 0, "time": 0},
          "mysql": {"count": 0, "time": 0},
          "postgresql": {"count": 0, "time": 0}
        },
        "100": {
          "r": {"count": 0, "time": 0},
          "tensorflow": {"count": 0, "time": 0},
          "mysql": {"count": 0, "time": 0},
          "postgresql": {"count": 0, "time": 0}
        },
        "1000": {
          "r": {"count": 0, "time": 0},
          "tensorflow": {"count": 0, "time": 0},
          "mysql": {"count": 0, "time": 0},
          "postgresql": {"count": 0, "time": 0}
        },
        "10000": {
          "r": {"count": 0, "time": 0},
          "tensorflow": {"count": 0, "time": 0},
          "mysql": {"count": 0, "time": 0},
          "postgresql": {"count": 0, "time": 0}
        },
        "100000": {
          "r": {"count": 0, "time": 0},
          "tensorflow": {"count": 0, "time": 0},
          "mysql": {"count": 0, "time": 0},
          "postgresql": {"count": 0, "time": 0}
        }
      }
    }

    # Iteriere über alle Zeilen der csv-Datei.
    for row in csvreader:
      # Überspringe die erste Zeile.
      if not row[0] == "language":
        # Füge die Laufzeit in das dictionary ein.
        count[row[1]][row[2]][row[0]]["count"] += 1
        count[row[1]][row[2]][row[0]]["time"] += float(row[3])

    # Durchlaufe das dictionary (Typ der Regression).
    for regression_type, obj1 in count.items():
      print("benchmarks for %s:\n" % regression_type)

      # Erzeuge ein neues dictionary, um eine Tabelle zu drucken.
      table = {"header": [""], "r": ["r"], "tensorflow": ["tensorflow"], "mysql": ["mysql"], "postgresql": ["postresql"]}

      # Iteriere über die im ersten dictionary enthalten dictionaries (Anzahl Datenpunkte).
      for number_datapoints, obj2 in obj1.items():
        # Ergänze die Anzahl der Datenpunkte als Spaltenbeschriftung.
        table["header"].append(str(number_datapoints))

        # Iteriere über die im zweiten dictionary enthalten dictionaries (Sprache).
        for language, data in obj2.items():
          # Füge die durchschnittliche Laufzeit in die Tabelle ein (falls Lafzeiten vorhanden sind).
          if data["count"] > 0:
            table[language].append(str(data["time"] / data["count"])[:10])
          else:
            table[language].append("          ")

      # Erzeuge die zu druckende Tabelle zeilenweise.
      print_table = [
        "|" + "-" * 89 + "|",
        "|" + "  %s            |  %s          |  %s         |  %s        |  %s       |  %s      " % tuple(table["header"]) + "|",
        "|" + "-" * 89 + "|",
        "|" + "  %s           |  %s  |  %s  |  %s  |  %s  |  %s  " % tuple(table["r"]) + "|",
        "|" + "-" * 89 + "|",
        "|" + "  %s  |  %s  |  %s  |  %s  |  %s  |  %s  " % tuple(table["tensorflow"]) + "|",
        "|" + "-" * 89 + "|",
        "|" + "  %s       |  %s  |  %s  |  %s  |  %s  |  %s  " % tuple(table["mysql"]) + "|",
        "|" + "-" * 89 + "|",
        "|" + "  %s   |  %s  |  %s  |  %s  |  %s  |  %s  " % tuple(table["postgresql"]) + "|",
        "|" + "-" * 89 + "|"
      ]

      # Drucke die Tabelle und eine Leerzeile danach.
      print(str.join("\n", print_table))
      print("\n")

    # Gib das dictionary mit allen aggregierten Laufzeiten zurück.
    return count

  # Erzeuge Funktion, um die Laufzeiten für eine bestimmte Art der Regression zu plotten.
  def plot(benchmarks, regression_type, plot_title):
    # Definiere ein Array mit den Anzahlen der Datenpunkte für den Plot.
    x = [10, 100, 1000, 10000, 100000]

    # Erzeuge ein dictionary mit leeren Arrays für die durchschnittlichen Laufzeiten.
    values = {
      "r": [],
      "tensorflow": [],
      "mysql": [],
      "postgresql": []
    }

    # Durchlaufe alle Sprachen.
    for language in ["r", "tensorflow", "mysql", "postgresql"]:
      # Durchlaufe die Anzahl der Datenpunkte.
      for number_datapoints in x:
        # Füge die durchschnittliche Laufzeit in das Array ein (falls Laufzeiten vorhanden sind).
        if benchmarks[regression_type][str(number_datapoints)][language]["count"] > 0:
          values[language].append(
            benchmarks[regression_type][str(number_datapoints)][language]["time"] /
            benchmarks[regression_type][str(number_datapoints)][language]["count"]
          )
        else:
          values[language].append(None)

    # Erzeuge den Plot.
    plt.loglog(x, values["r"], "r-", label="R")
    plt.loglog(x, values["tensorflow"], "y-", label="TensorFlow")
    plt.loglog(x, values["mysql"], "b-", label="MySQL")
    plt.loglog(x, values["postgresql"], "g-", label="PostgreSQL")
    plt.title(plot_title)
    plt.legend()
    plt.show()

  def main(argv):
    # Erzeuge default-Wert, ob Plots erstellt werden sollen.
    print_plots = True

    # Überschreibe default-Wert, falls ein entsprechendes Argument übergeben wurde.
    if len(argv) == 2:
      if argv[1] == "-":
        print_plots = False

    # Berechne und drucke die Benchmarks.
    benchmarks = calculate_benchmarks()

    # Plote die Benchmarks (falls gewüscht).
    if print_plots:
      plot(benchmarks, "simple linear regression", "Einfache lineare Regression")
      plot(benchmarks, "multiple linear regression", "Multiple lineare Regression")
      plot(benchmarks, "logistic regression", "Logistische Regression")

  # Führe die main-Funktion aus.
  if __name__ == "__main__":
    main(sys.argv)
\end{minted}
