\chapter{Python-Skript zum Generieren der Beispieldaten}
\label{appendix:A}

\begin{minted}[linenos,breaklines,frame=single]{python}
  import sys
  import random
  import math

  def outputCsv(data):
    output = ["%s,%s,%s,%s" % (
      "age",
      "purchases",
      "money",
      "prime"
    )]

    for datapoint in data:
      output.append("%s,%s,%s,%a" % (
        datapoint["age"],
        datapoint["purchases"],
        datapoint["money"],
        datapoint["prime"]
      ))

    f = open("sample.csv", "w")
    f.write("\n".join(output))

    return


  def outputSql(data):
    output = [
      "DROP TABLE IF EXISTS sample;",
      "",
      "CREATE TABLE sample (",
      "  age INTEGER,",
      "  purchases INTEGER,",
      "  money INTEGER,",
      "  prime INTEGER",
      ");",
      "",
      "INSERT INTO sample (%s,%s,%s,%s) VALUES" % (
        "age",
        "purchases",
        "money",
        "prime"
      )
    ]

    for datapoint in data:
      output.append("(%s,%s,%s,%s)," % (
        datapoint["age"],
        datapoint["purchases"],
        datapoint["money"],
        datapoint["prime"]
      ))

    output[-1] = output[-1][:-1] + ";"
    f = open("sample.sql", "w")
    f.write("\n".join(output))

    return


  def main(argv):
    if len(argv) < 2:
      print("Please provide number of datapoints that shall be generated.")
      return

    data = []

    for i in range(0, int(argv[1])):
      age = int(max(random.normalvariate(25, 10) + 10, 18))
      purchases = int(max(random.normalvariate(10, 10), 1))
      money = int(max(purchases * 25 + random.normalvariate(0, (math.log(purchases) + 1) * 12), 0.01) * 100)
      if random.uniform(0, 1) > math.exp(0.2 * purchases - 2) / (1 + math.exp(0.2 * purchases - 2)):
        prime = 0
      else:
        prime = 1

      data.append({
        "age": age,
        "purchases": purchases,
        "money": money,
        "prime": prime
      })

    outputCsv(data)
    outputSql(data)
    return

  if __name__ == "__main__":
    main(sys.argv)
\end{minted}

\chapter{R-Skripte}
\label{appendix:B}

\section{Einfache lineare Regression}
\label{appendix:B:1}

\begin{minted}[linenos,breaklines,frame=single]{r}
  args = commandArgs(trailingOnly = TRUE)

  n <- 100000
  plot <- TRUE
  if (length(args) == 1) {
    if (substr(args[1], 1, 1) == "-") {
      plot <- FALSE
    } else {
      n = strtoi(args[1])
    }
  }
  if (length(args) == 2) {
    if (substr(args[1], 1, 1) == "-") {
      n = strtoi(args[2])
    } else {
      n = strtoi(args[1])
    }
    plot <- FALSE
  }

  start_time <- Sys.time()

  data <- head(read.csv2("./data/sample.csv", sep = ",", header = TRUE), n)

  modell <- as.formula("money ~ purchases")
  slr <- lm(modell, data = data)

  end_time <- Sys.time()

  print(slr)
  print(end_time - start_time)

  if (plot) {
    xmin <- min(data$purchases)
    xmax <- max(data$purchases)

    ymin <- min(data$money)
    ymax <- max(data$money)

    b0 <- coef(slr["coefficients"])[1]
    b1 <- coef(slr["coefficients"])[2]

    xplot <- c(xmin - 1, xmax + 1)
    yplot <- c(b0 + (xmin - 1) * b1, b0 + (xmax + 1) * b1)

    plot(
      c(xmin - 1, xmax + 1),
      c(ymin - 1, ymax + 1),
      type = "n",
      xlab = "purchases",
      ylab = "money",
      main = "Einfache lineare Regression",
      sub = paste("money = ", b0, "+", b1, "* purchases"),
      col.sub = "darkgray"
    )
    lines(
      data$purchases,
      data$money,
      type="p"
    )
    lines(
      xplot,
      yplot,
      col = "red",
      lwd = 2
    )
  }
\end{minted}

\section{Multiple lineare Regression}
\label{appendix:B:2}

\begin{minted}[linenos,breaklines,frame=single]{r}
  args = commandArgs(trailingOnly = TRUE)

  n = 100000
  if (length(args) > 0) {
    n = strtoi(args[1])
  }

  start_time <- Sys.time()

  data <- head(read.csv2("./data/sample.csv", sep = ",", header = TRUE), n)

  modell <- as.formula("money ~ purchases + age")
  mlr <- lm(modell, data = data)

  end_time <- Sys.time()

  print(mlr)
  print(end_time - start_time)
\end{minted}

\section{Logistische Regression}
\label{appendix:B:3}

\begin{minted}[linenos,breaklines,frame=single]{r}
  args = commandArgs(trailingOnly = TRUE)

  n <- 100000
  plot <- TRUE
  if (length(args) == 1) {
    if (substr(args[1], 1, 1) == "-") {
      plot <- FALSE
    } else {
      n = strtoi(args[1])
    }
  }
  if (length(args) == 2) {
    if (substr(args[1], 1, 1) == "-") {
      n = strtoi(args[2])
    } else {
      n = strtoi(args[1])
    }
    plot <- FALSE
  }

  start_time <- Sys.time()

  data <- head(read.csv2("./data/sample.csv", sep = ",", header = TRUE), n)

  modell <- as.formula("prime ~ money")
  logit <- glm(modell, family = binomial, data = data)

  end_time <- Sys.time()

  print(logit)
  print(end_time - start_time)

  if (plot) {
    xmin <- min(data$money)
    xmax <- max(data$money)

    logitFunction <- function(x){
      b0 <- coef(logit["coefficients"])[1]
      b1 <- coef(logit["coefficients"])[2]
      c <- b0 + x * b1
      return(exp(c) / (1 + exp(c)))
    }

    xplot <- seq(xmin - 1, xmax + 1, 1000)
    yplot <- logitFunction(xplot)

    plot(
      c(xmin - 1, xmax + 1),
      c(-0.2, 1.2),
      type = "n",
      xlab = "money",
      ylab = "prime",
      main = "Logistische Regression"
    )
    lines(
      data$money,
      data$prime,
      type="p"
    )
    lines(
      xplot,
      yplot,
      col = "red",
      lwd = 2
    )
  }
\end{minted}

\chapter{TensorFlow-Skripte}
\label{appendix:C}

\section{Einfache lineare Regression}
\label{appendix:C:1}

\begin{minted}[linenos,breaklines,frame=single]{python}
  import numpy as np
  import tensorflow as tf
  import os.path as p
  import csv
  import sys
  import matplotlib.pyplot as plt
  from time import time

  def get_data(n_samples):
    filename = p.abspath(p.join(p.dirname(p.realpath(__file__)), "..", "data", "sample.csv"))
    csvfile = open(filename, newline="")
    x = []
    x_plot = []
    y = []

    csvreader = csv.reader(csvfile, delimiter=",", quotechar="|")
    for row in csvreader:
      if not row[0] == "age":
        x.append([int(row[1])])
        x_plot.append(int(row[1]))
        y.append(int(row[2]))

    return (np.array(x[:n_samples]), x_plot[:n_samples], np.transpose([y[:n_samples]]))

  def main(argv):
    datapoint_size = 100000
    plot = True
    if len(argv) == 2:
      if argv[1] == "-":
        plot = False
      else:
        datapoint_size = int(argv[1])
    elif len(argv) == 3:
      plot = False
      if argv[1] == "-":
        datapoint_size = int(argv[2])
      else:
        datapoint_size = int(argv[1])

    start_time = time()

    steps = 2000
    if datapoint_size <= 10:
      learn_rate = 0.0076
    elif datapoint_size <= 100:
      learn_rate = 0.0064
    elif datapoint_size <= 1000:
      learn_rate = 0.0056
    elif datapoint_size <= 10000:
      learn_rate = 0.0054
    elif datapoint_size <= 100000:
      learn_rate = 0.0054

    x = tf.placeholder(tf.float32, [None, 1])
    y = tf.placeholder(tf.float32, [None, 1])
    alpha = tf.Variable(tf.zeros([1]))
    beta = tf.Variable(tf.zeros([1, 1]))
    y_calc = tf.matmul(x, beta) + alpha

    cost = tf.reduce_mean(tf.square(y - y_calc))
    train_step = tf.train.GradientDescentOptimizer(learn_rate).minimize(cost)

    (all_xs, plot_xs, all_ys) = get_data(datapoint_size)

    sess = tf.Session()
    init = tf.global_variables_initializer()
    sess.run(init)

    for i in range(steps):
      feed = { x: all_xs, y: all_ys }
      sess.run(train_step, feed_dict=feed)

    (curr_alpha, curr_beta, curr_cost) = sess.run([alpha, beta, cost], feed_dict=feed)

    end_time = time()

    print("alpha:  %f" % curr_alpha)
    print("beta:   %f" % curr_beta)
    print("cost:   %f" % curr_cost)
    print("")
    print("time elapsed:  %f sec" % (end_time - start_time))

    if plot:
      plt.plot(plot_xs, all_ys, "ro", label="Original data")
      plt.plot(plot_xs, curr_beta * all_xs + curr_alpha , label="Fitted line")
      plt.legend()
      plt.show()

  if __name__ == "__main__":
    main(sys.argv)
\end{minted}

\section{Multiple lineare Regression}
\label{appendix:C:2}

\begin{minted}[linenos,breaklines,frame=single]{python}
  import numpy as np
  import tensorflow as tf
  import os.path as p
  import csv
  import sys
  from time import time

  def get_data(n_samples):
    filename = p.abspath(p.join(p.dirname(p.realpath(__file__)), "..", "data", "sample.csv"))
    csvfile = open(filename, newline="")
    x = []
    y = []

    csvreader = csv.reader(csvfile, delimiter=",", quotechar="|")
    for row in csvreader:
      if not row[0] == "age":
        x.append([int(row[1]), int(row[0])])
        y.append(int(row[2]))

    return (np.array(x[:n_samples]), np.transpose([y[:n_samples]]))

  def main(argv):
    datapoint_size = 100000
    if len(argv) == 2:
      datapoint_size = int(argv[1])

    start_time = time()

    steps = 50000
    if datapoint_size <= 10:
      learn_rate = 0.00093
    elif datapoint_size <= 100:
      learn_rate = 0.00078
    elif datapoint_size <= 1000:
      learn_rate = 0.0007
    elif datapoint_size <= 10000:
      learn_rate = 0.00071
    elif datapoint_size <= 100000:
      learn_rate = 0.00071

    x = tf.placeholder(tf.float32, [None, 2])
    y = tf.placeholder(tf.float32, [None, 1])
    alpha = tf.Variable(tf.zeros([1]))
    beta = tf.Variable(tf.zeros([2, 1]))
    y_calc = tf.matmul(x, beta) + alpha

    cost = tf.reduce_mean(tf.square(y - y_calc))
    train_step = tf.train.GradientDescentOptimizer(learn_rate).minimize(cost)

    (all_xs, all_ys) = get_data(datapoint_size)

    sess = tf.Session()
    init = tf.global_variables_initializer()
    sess.run(init)

    for i in range(steps):
      feed = { x: all_xs, y: all_ys }
      sess.run(train_step, feed_dict=feed)

    (curr_alpha, curr_beta, curr_cost) = sess.run([alpha, beta, cost], feed_dict=feed)

    end_time = time()

    print("alpha:           %f" % curr_alpha)
    print("beta_purchases:  %f" % curr_beta[0])
    print("beta_age:        %f" % curr_beta[1])
    print("cost:            %f" % curr_cost)
    print("")
    print("time elapsed: %f sec" % (end_time - start_time))

  if __name__ == "__main__":
    main(sys.argv)
\end{minted}

\section{Logistische Regression}
\label{appendix:C:3}

\begin{minted}[linenos,breaklines,frame=single]{python}
  import numpy as np
  import tensorflow as tf
  import os.path as p
  import csv
  import sys
  import matplotlib.pyplot as plt
  from time import time

  def get_data(n_samples):
    filename = p.abspath(p.join(p.dirname(p.realpath(__file__)), "..", "data", "sample.csv"))
    csvfile = open(filename, newline="")
    x = []
    x_plot = []
    y = []

    csvreader = csv.reader(csvfile, delimiter=",", quotechar="|")
    for row in csvreader:
      if not row[0] == "age":
        x.append([int(row[2])])
        x_plot.append(int(row[2]))
        y.append(int(row[3]))

    return (np.array(x[:n_samples]), x_plot[:n_samples], np.transpose([y[:n_samples]]))

  def main(argv):
    datapoint_size = 100000
    plot = True
    if len(argv) == 2:
      if argv[1] == "-":
        plot = False
      else:
        datapoint_size = int(argv[1])
    elif len(argv) == 3:
      plot = False
      if argv[1] == "-":
        datapoint_size = int(argv[2])
      else:
        datapoint_size = int(argv[1])

    start_time = time()

    steps = 1000
    if datapoint_size == 10:
      learn_rate = 1
    elif datapoint_size == 100:
      learn_rate = 0.1
    elif datapoint_size == 1000:
      learn_rate = 0.01
    elif datapoint_size == 10000:
      learn_rate = 0.001
    elif datapoint_size == 100000:
      learn_rate = 0.0001

    x = tf.placeholder(tf.float32, [None, 1])
    y = tf.placeholder(tf.float32, [None, 1])
    alpha = tf.Variable(tf.zeros([1]))
    beta = tf.Variable(tf.zeros([1, 1]))
    y_calc = 1 / (1 + tf.exp(- tf.matmul(x, beta) - alpha))

    cost = - tf.reduce_sum(
      tf.log(
        y * y_calc +
        (1 - y) * (1 - y_calc)
      )
    )
    train_step = tf.train.GradientDescentOptimizer(learn_rate).minimize(cost)

    (all_xs, plot_xs, all_ys) = get_data(datapoint_size)

    min_x = min(all_xs)
    max_x = max(all_xs)
    all_xs = (all_xs - min_x) / (max_x - min_x)

    sess = tf.Session()
    init = tf.global_variables_initializer()
    sess.run(init)

    for i in range(steps):
      feed = { x: all_xs, y: all_ys }
      sess.run(train_step, feed_dict=feed)

    (curr_alpha, curr_beta, curr_cost) = sess.run([alpha, beta, cost], feed_dict=feed)

    curr_beta = curr_beta / (max_x - min_x)
    curr_alpha = curr_alpha - curr_beta * min_x

    end_time = time()

    print("alpha:  %f" % curr_alpha)
    print("beta:   %f" % curr_beta)
    print("cost:   %f" % curr_cost)
    print("")
    print("time elapsed: %f sec" % (end_time - start_time))

    if plot:
      all_xs = all_xs * (max_x - min_x) + min_x
      plot_ys = 1 / (1 + np.exp(- curr_beta * all_xs - curr_alpha))
      plot_order = np.argsort(plot_xs)
      plt.plot(plot_xs, all_ys, "ro", label="Original data")
      plt.plot(np.array(plot_xs)[plot_order], np.array(plot_ys)[plot_order], label="Fitted line")
      plt.legend()
      plt.show()

  if __name__ == "__main__":
    main(sys.argv)
\end{minted}

\chapter{MySQL-Skripte}
\label{appendix:D}

\section{Einfache lineare Regression}
\label{appendix:D:1}

\begin{minted}[linenos,breaklines,frame=single]{sql}
  -- drop possibly existing procedures
  DROP PROCEDURE IF EXISTS simple_linear_regression;

  DELIMITER ;;

  -- main procedure for regression analysis
  CREATE PROCEDURE `simple_linear_regression`(IN number_datapoints INT(11))
  BEGIN

  -- declare variables
  DECLARE purchases_mean DECIMAL(40, 20);
  DECLARE money_mean DECIMAL(40, 20);
  DECLARE alpha DECIMAL(40, 20);
  DECLARE beta DECIMAL(40, 20);

  -- create temporary table with datapoints
  DROP TEMPORARY TABLE IF EXISTS datapoints;
  CREATE TEMPORARY TABLE datapoints (
    purchases INT(11),
    money INT(11)
  );
  INSERT INTO datapoints
  SELECT purchases, money
  FROM sample
  LIMIT number_datapoints;

  -- calculate means
  SET purchases_mean = (
    SELECT AVG(purchases)
    FROM datapoints
  );
  SET money_mean = (
    SELECT AVG(money)
    FROM datapoints
  );

  -- calculate beta
  SET beta = (
    SELECT SUM((purchases - purchases_mean) * (money - money_mean))
    FROM datapoints
  );
  SET beta = beta / (
    SELECT SUM(POWER(purchases - purchases_mean, 2))
    FROM datapoints
  );

  -- calculate alpha
  SET alpha = money_mean - (beta * purchases_mean);

  -- print parameters alpha and beta
  SELECT 'alpha' AS `variable`, alpha AS `value`
  UNION
  SELECT 'beta' AS `variable`, beta AS `value`;

  DROP TEMPORARY TABLE IF EXISTS datapoints;

  END;;

  DELIMITER ;
\end{minted}

\section{Multiple lineare Regression}
\label{appendix:D:2}

\begin{minted}[linenos,breaklines,frame=single]{sql}
  -- drop possibly existing procedures
  DROP PROCEDURE IF EXISTS multiple_linear_regression;

  DELIMITER ;;

  -- main procedure for regression analysis
  CREATE PROCEDURE multiple_linear_regression(IN number_datapoints INT(11))
  BEGIN

  -- declare variables
  DECLARE m INT(11);
  DECLARE n INT(11);
  DECLARE counter_1 INT(11);
  DECLARE counter_2 INT(11);
  DECLARE counter_3 INT(11);
  DECLARE pivot DECIMAL(40, 20);

  -- set matrix dimensions
  SET m = number_datapoints;
  SET n = 3;

  -- drop temporary tables if existing
  DROP TEMPORARY TABLE IF EXISTS matrix_X;
  DROP TEMPORARY TABLE IF EXISTS matrix_transposed;
  DROP TEMPORARY TABLE IF EXISTS matrix_product_1;
  DROP TEMPORARY TABLE IF EXISTS matrix_inverse;
  DROP TEMPORARY TABLE IF EXISTS matrix_product_2;
  DROP TEMPORARY TABLE IF EXISTS matrix_y;
  DROP TEMPORARY TABLE IF EXISTS matrix_result;

  -- create temporary tables
  CREATE TEMPORARY TABLE matrix_X (
    `row` INT(11),
    `column` INT(11),
    `value` DECIMAL(40, 20)
  );
  CREATE TEMPORARY TABLE matrix_transposed (
    `row` INT(11),
    `column` INT(11),
    `value` DECIMAL(40, 20)
  );
  CREATE TEMPORARY TABLE matrix_product_1 (
    `row` INT(11),
    `column` INT(11),
    `value` DECIMAL(40, 20)
  );
  CREATE TEMPORARY TABLE matrix_inverse (
    `row` INT(11),
    `column` INT(11),
    `value` DECIMAL(40, 20)
  );
  CREATE TEMPORARY TABLE matrix_product_2 (
    `row` INT(11),
    `column` INT(11),
    `value` DECIMAL(40, 20)
  );
  CREATE TEMPORARY TABLE matrix_y (
    `row` INT(11),
    `column` INT(11),
    `value` DECIMAL(40, 20)
  );
  CREATE TEMPORARY TABLE matrix_result (
    `row` INT(11),
    `column` INT(11),
    `value` DECIMAL(40, 20)
  );

  -- insert constant values in matrix_X
  SET @id = 0;

  INSERT INTO matrix_X
  SELECT
    @id := (@id + 1) AS `row`,
    1 AS `column`,
    1 AS `value`
  FROM sample
  LIMIT number_datapoints;

  -- insert values for purchases in matrix_X
  SET @id = 0;

  INSERT INTO matrix_X
  SELECT
    @id := (@id + 1) AS `row`,
    2 AS `column`,
    purchases AS `value`
  FROM sample
  LIMIT number_datapoints;

  -- insert values for age in matrix_X
  SET @id = 0;

  INSERT INTO matrix_X
  SELECT
    @id := (@id + 1) AS `row`,
    3 AS `column`,
    age AS `value`
  FROM sample
  LIMIT number_datapoints;

  -- insert values for money in matrix_y
  SET @id = 0;

  INSERT INTO matrix_y
  SELECT
    @id := (@id + 1) AS `row`,
    1 AS `column`,
    money AS `value`
  FROM sample
  LIMIT number_datapoints;

  -- calculate matrix_transposed
  INSERT INTO matrix_transposed
  SELECT
    `column` AS `row`,
    `row` AS `column`,
    `value` AS `value`
  FROM matrix_X;

  -- calculate Matrix_Product1
  SET counter_1 = 1;

  WHILE counter_1 <= n DO

    SET counter_2 = 1;

    WHILE counter_2 <= n DO

      INSERT INTO matrix_product_1 VALUES (
        counter_1,
        counter_2,
        (
          SELECT SUM(T1.`value` * T2.`value`)
          FROM (
            SELECT * FROM matrix_X WHERE `column` = counter_2
          ) T1
          JOIN (
            SELECT * FROM matrix_transposed WHERE `row` = counter_1
          ) T2
          ON T2.`column` = T1.`row`
        )
      );

      SET counter_2 = counter_2 + 1;

    END WHILE;

    SET counter_1 = counter_1 + 1;

  END WHILE;

  -- calculate matrix_inverse
  INSERT INTO matrix_inverse
  SELECT *
  FROM matrix_product_1;

  SET counter_1 = 0;

  WHILE counter_1 < n DO

    SET counter_1 = counter_1 + 1;

    DROP TEMPORARY TABLE IF EXISTS pivot_row;
    CREATE TEMPORARY TABLE pivot_row (
        `column` INT(11),
        `value` DECIMAL(40, 20)
    );

    INSERT INTO pivot_row
    SELECT `column`, `value`
    FROM matrix_inverse
    WHERE `row` = counter_1;

    SET pivot = (
        SELECT `value`
        FROM matrix_inverse
        WHERE `row` = counter_1 AND `column` = counter_1
    );

    UPDATE matrix_inverse
    SET `value` = `value` / pivot
    WHERE `row` = counter_1 AND `column` <> counter_1;

    UPDATE matrix_inverse
    SET `value` = - `value` / pivot
    WHERE `row` <> counter_1 AND `column` = counter_1;

    SET counter_2 = 1;

    WHILE counter_2 <= n DO

      IF counter_2 <> counter_1 THEN

        SET counter_3 = 1;

        WHILE counter_3 <= n DO

          IF counter_3 <> counter_1 THEN

            SET pivot = (
              SELECT `value`
              FROM pivot_row
              WHERE `column` = counter_3
            ) * (
              SELECT `value`
              FROM matrix_inverse
              WHERE `row` = counter_2 AND `column` = counter_1
            );

            UPDATE matrix_inverse
            SET `value` = `value` + pivot
            WHERE `row` = counter_2 AND `column` = counter_3;

          END IF;

          SET counter_3 = counter_3 + 1;

        END WHILE;

      END IF;

      SET counter_2 = counter_2 + 1;

    END WHILE;

    UPDATE matrix_inverse
    SET `value` = 1 / `value`
    WHERE `row` = counter_1 AND `column` = counter_1;

  END WHILE;

  -- calculate matrix_product_2
  SET counter_1 = 1;

  WHILE counter_1 <= n DO

    INSERT INTO matrix_product_2 VALUES (
      counter_1,
      1,
      (
        SELECT SUM(T1.`value` * T2.`value`)
        FROM (
          SELECT * FROM matrix_y
        ) T1
        JOIN (
          SELECT * FROM matrix_transposed WHERE `row` = counter_1
        ) T2
        ON T2.`column` = T1.`row`
      )
    );

    SET counter_1 = counter_1 + 1;

  END WHILE;

  -- calculate matrix_result

  SET counter_1 = 1;

  WHILE counter_1 <= n DO

    INSERT INTO matrix_result VALUES (
      counter_1,
      1,
      (
        SELECT SUM(T1.`value` * T2.`value`)
        FROM (
          SELECT * FROM matrix_product_2
        ) T1
        JOIN (
          SELECT * FROM matrix_inverse WHERE `row` = counter_1
        ) T2
        ON T2.`column` = T1.`row`
      )
    );

    SET counter_1 = counter_1 + 1;

  END WHILE;

  SELECT
    CASE row
      WHEN 1 THEN 'alpha'
      WHEN 2 THEN 'beta_purchases'
      WHEN 3 THEN 'beta_age'
    END AS `variable`,
    value
  FROM matrix_result;

  END;;

  DELIMITER ;
\end{minted}

\section{Logistische Regression}
\label{appendix:D:3}

\begin{minted}[linenos,breaklines,frame=single]{sql}
  DROP PROCEDURE IF EXISTS calculate_gradient;
  DROP PROCEDURE IF EXISTS calculate_new_parameters;
  DROP PROCEDURE IF EXISTS calculate_logit;
  DROP PROCEDURE IF EXISTS are_new_parameters_better;
  DROP PROCEDURE IF EXISTS logistic_regression;

  DELIMITER ;;
  -- this procedure calculates the logits for current parameter values
  CREATE PROCEDURE `calculate_logit`()
  BEGIN

  DECLARE alpha_old DECIMAL(65, 30);
  DECLARE alpha_new DECIMAL(65, 30);

  SET alpha_old = (
    SELECT old
    FROM parameters
    WHERE variable = 'alpha'
  );

  SET alpha_new = (
    SELECT new
    FROM parameters
    WHERE variable = 'alpha'
  );

  DELETE FROM logits;

  INSERT INTO logits
    SELECT
      d.id,
      1 / (1 + exp(- SUM(d.value * p.old) - alpha_old)) AS `old`,
      1 / (1 + exp(- SUM(d.value * p.new) - alpha_new)) AS `new`
    FROM datapoints d
    JOIN parameters p ON p.variable = d.variable
    GROUP BY d.id;

  END;;

  -- this procedure calculates the gradient for the current parameter values
  CREATE PROCEDURE `calculate_gradient`()
  BEGIN

  DELETE FROM gradient;

  -- calculate gradient for alpha
  INSERT INTO gradient
  SELECT 'alpha' AS `variable`, SUM(bv.value - l.old) AS `value`
  FROM logits l
  JOIN binary_values bv ON bv.id = l.id;

  -- calculate other gradients
  INSERT INTO gradient
  SELECT d.variable, SUM(d.value * (bv.value - l.old)) AS `value`
  FROM logits l
  JOIN binary_values bv ON bv.id = l.id
  JOIN datapoints d ON d.id = l.id
  GROUP BY d.variable;

  END;;

  -- this procedure calculates the new parameters
  CREATE PROCEDURE `calculate_new_parameters`(IN step DECIMAL(65, 30))
  BEGIN

  UPDATE parameters
  JOIN gradient ON gradient.variable = parameters.variable
  SET parameters.new = parameters.old + step * gradient.value;

  END;;

  -- this procedure calculates the log likelihood function for current parameter values
  -- and states if the new parameters are really better
  CREATE PROCEDURE `are_new_parameters_better`(OUT better INT(1))
  BEGIN

  SET better = (
    SELECT
      SUM(LOG(bv.value * l.new + (1 - bv.value) * (1 - l.new))) >
      SUM(LOG(bv.value * l.old + (1 - bv.value) * (1 - l.old)))
    FROM logits l
    JOIN binary_values bv ON bv.id = l.id
  );

  END;;

  -- main procedure for execution
  CREATE PROCEDURE `logistic_regression`(IN number_datapoints INT(11), IN rounds INT(11), step DECIMAL(65, 30))
  BEGIN

  DECLARE min INT(11);
  DECLARE max INT(11);
  DECLARE transform DECIMAL(65, 30);
  DECLARE better INT(1);
  DECLARE counter INT(11);

  -- create a temporary table for the data
  DROP TEMPORARY TABLE IF EXISTS datapoints;
  CREATE TEMPORARY TABLE datapoints (
    id INT(11),
    variable VARCHAR(32),
    value DECIMAL(65, 30),
    PRIMARY KEY (id, variable)
  );

  -- calculate min and max values for money
  SET min = (SELECT MIN(money) FROM sample);
  SET max = (SELECT MAX(money) FROM sample);

  -- insert all linear transformed values for column 'money' into data
  SET @counter = 0;
  INSERT INTO datapoints
  SELECT
    @counter := @counter + 1 AS `id`,
    'beta_money' AS `variable`,
    (money - min) / (max - min) AS `value`
  FROM sample
  LIMIT number_datapoints;

  -- create temporary table for binary variable
  DROP TEMPORARY TABLE IF EXISTS binary_values;
  CREATE TEMPORARY TABLE binary_values (
    id INT(11),
    value INT(1),
    PRIMARY KEY (id)
  );

  -- insert all values for column 'prime' into binary_values
  SET @counter = 0;
  INSERT INTO binary_values
  SELECT
    @counter := @counter + 1 AS `id`,
    prime AS `value`
  FROM sample
  LIMIT number_datapoints;

  -- create temporary table for parameters
  DROP TEMPORARY TABLE IF EXISTS parameters;
  CREATE TEMPORARY TABLE parameters (
    variable VARCHAR(32),
    old DECIMAL(65, 30),
    new DECIMAL(65, 30),
    PRIMARY KEY (variable)
  );

  -- set initial parameters
  INSERT INTO parameters VALUES
    ('alpha', 0, 0),
    ('beta_money', 0, 0);

  -- create temporary table for logits
  DROP TEMPORARY TABLE IF EXISTS logits;
  CREATE TEMPORARY TABLE logits (
    id INT(11),
    old DECIMAL(65, 30),
    new DECIMAL(65, 30),
    PRIMARY KEY (id)
  );

  -- insert initial values into logit table
  CALL calculate_logit();

  -- create temporary table for gradient
  DROP TEMPORARY TABLE IF EXISTS gradient;
  CREATE TEMPORARY TABLE gradient (
    variable VARCHAR(32),
    value DECIMAL(65, 30),
    PRIMARY KEY (variable)
  );

  -- insert variables in gradient table
  INSERT INTO gradient VALUES
    ('alpha', 0),
    ('beta_money', 0);

  -- loop
  SET counter = 0;
  WHILE counter < rounds AND step > 0.000000000000000000000000000001 DO

    CALL calculate_gradient();
    CALL calculate_new_parameters(step);
    CALL calculate_logit();

    WHILE (
      SELECT
        SUM(LOG(bv.value * l.new + (1 - bv.value) * (1 - l.new))) >
        SUM(LOG(bv.value * l.old + (1 - bv.value) * (1 - l.old)))
      FROM logits l
      JOIN binary_values bv ON bv.id = l.id
    ) = 0 AND step > 0.000000000000000000000000000001 DO

      SET step = step / 2;
      CALL calculate_new_parameters(step);
      CALL calculate_logit();

    END WHILE;

    UPDATE parameters
    SET parameters.old = parameters.new;

    UPDATE logits
    SET logits.old = logits.new;

    SET counter = counter + 1;

  END WHILE;

  UPDATE parameters
  SET old = old / (max - min)
  WHERE variable = 'beta_money';

  SET transform = (SELECT old FROM parameters WHERE variable = 'beta_money');

  UPDATE parameters
  SET old = old - transform * min
  WHERE variable = 'alpha';

  SELECT variable, old AS `value`
  FROM parameters;

  END;;
  DELIMITER ;
\end{minted}

\chapter{PostgreSQL-Skripte}
\label{appendix:E}

\section{Einfache lineare Regression}
\label{appendix:E:1}

\begin{minted}[linenos,breaklines,frame=single]{sql}
  -- main procedure for regression analysis
  CREATE OR REPLACE FUNCTION simple_linear_regression(number_datapoints INTEGER)
  RETURNS TABLE (
    variable VARCHAR(50),
    value NUMERIC(65, 30)
  ) AS $$
  BEGIN

  DROP TABLE IF EXISTS datapoints;
  CREATE TEMPORARY TABLE datapoints (
    purchases INTEGER,
    money INTEGER
  );

  INSERT INTO datapoints
  SELECT purchases, money
  FROM sample
  LIMIT number_datapoints;

  RETURN QUERY
  WITH
    means AS (
      SELECT
        AVG(purchases) AS mean_purchases,
        AVG(money) AS mean_money
      FROM datapoints
    ),
    sums AS (
      SELECT
        SUM((purchases - (SELECT mean_purchases FROM means)) * (money - (SELECT mean_money FROM means))) AS nominator,
        SUM(POWER(purchases - (SELECT mean_purchases FROM means), 2)) AS denominator
      FROM datapoints
    ),
    beta AS (
      SELECT
        'beta'::VARCHAR(50) AS variable,
        nominator / denominator AS value
      FROM sums
    ),
    alpha AS (
      SELECT
        'alpha'::VARCHAR(50) AS variable,
        mean_money - (SELECT beta.value FROM beta) * mean_purchases AS value
      FROM means
    )
  SELECT *
  FROM alpha
  UNION
  SELECT *
  FROM beta;

  END;
  $$ LANGUAGE plpgsql;
\end{minted}

\section{Multiple lineare Regression}
\label{appendix:E:2}

\begin{minted}[linenos,breaklines,frame=single]{sql}
  -- procedure for transposing a matrix
  CREATE OR REPLACE FUNCTION matrix_transpose(a NUMERIC(40, 20)[][])
  RETURNS NUMERIC(40, 20)[][] AS $$
  DECLARE
    rows_a INTEGER := array_length(a, 1);
    columns_a INTEGER := array_length(a, 2);
    i INTEGER;
    j INTEGER;
    c NUMERIC(40, 20)[][];
    new_row NUMERIC(40, 20)[];
  BEGIN

  i := 1;
  WHILE i <= columns_a LOOP

    j := 1;
    WHILE j <= rows_a LOOP
      new_row[j] := a[j][i];
      j := j + 1;
    END LOOP;

    c := array_cat(c, array[new_row]);
    i := i + 1;
  END LOOP;

  RETURN c;

  END;
  $$ LANGUAGE plpgsql;

  -- procedure for matrix multiplication
  CREATE OR REPLACE FUNCTION matrix_multiplication(a NUMERIC(40, 20)[][], b NUMERIC(40, 20)[][])
  RETURNS NUMERIC(40, 20)[][] AS $$
  DECLARE
    rows_a INTEGER := array_length(a, 1);
    columns_a INTEGER := array_length(a, 2);
    columns_b INTEGER := array_length(b, 2);
    new_row NUMERIC(40, 20)[];
    c NUMERIC(40, 20)[][];
    counter_1 INTEGER;
    counter_2 INTEGER;
    counter_3 INTEGER;
  BEGIN

  counter_1 := 1;
  WHILE counter_1 <= rows_a LOOP

    counter_2 := 1;
    WHILE counter_2 <= columns_b LOOP

      new_row[counter_2] := 0;
      counter_3 := 1;
      WHILE counter_3 <= columns_a LOOP
        new_row[counter_2] := new_row[counter_2] + a[counter_1][counter_3] * b[counter_3][counter_2];
        counter_3 := counter_3 + 1;
      END LOOP;

      counter_2 := counter_2 + 1;
    END LOOP;

    c := array_cat(c, array[new_row]);
    counter_1 := counter_1 + 1;
  END LOOP;

  RETURN c;

  END;
  $$ LANGUAGE plpgsql;

  -- main procedure for matrix inversion
  CREATE OR REPLACE FUNCTION matrix_inversion(a NUMERIC(40, 20)[][])
  RETURNS NUMERIC(40, 20)[][] AS $$
  DECLARE
    n INTEGER := array_length(a, 1);
    p INTEGER := 0;
    i INTEGER;
    j INTEGER;
    c NUMERIC(40, 20)[][] := a;
    o NUMERIC(40, 20)[][];
  BEGIN

  WHILE p < n LOOP

    p := p + 1;

    o := c;

    j := 1;
    WHILE j <= n LOOP
      IF j <> p THEN
        c[p][j] := c[p][j] / c[p][p];
      END IF;
      j := j + 1;
    END LOOP;

    i := 1;
    WHILE i <= n LOOP
      IF i <> p THEN
        c[i][p] := - c[i][p] / c[p][p];
      END IF;
      i := i + 1;
    END LOOP;

    i := 1;
    WHILE i <= n LOOP
      IF i <> p THEN
        j := 1;
        WHILE j <= n LOOP
          IF j <> p THEN
            c[i][j] := c[i][j] + o[p][j] * c[i][p];
          END IF;
          j := j + 1;
        END LOOP;
      END IF;
      i := i + 1;
    END LOOP;

    c[p][p] := 1 / c[p][p];

  END LOOP;

  RETURN c;

  END;
  $$ LANGUAGE plpgsql;

  -- main procedure for regression analysis
  CREATE OR REPLACE FUNCTION multiple_linear_regression(number_datapoints INTEGER)
  RETURNS TABLE (
    variable VARCHAR(50),
    value NUMERIC(65, 30)
  ) AS $$
  DECLARE
    x INTEGER[][]:= (
      SELECT ARRAY(
        SELECT ARRAY[1, purchases, age]
        FROM sample
        LIMIT number_datapoints
      )
    );
    y INTEGER[]:= (
      SELECT ARRAY(
        SELECT ARRAY[money]
        FROM sample
        LIMIT number_datapoints
      )
    );
    b NUMERIC[][];
  BEGIN

  b := matrix_multiplication(
    matrix_inversion(
      matrix_multiplication(
        matrix_transpose(x),
        x
      )
    ),
    matrix_multiplication(
      matrix_transpose(x),
      y
    )
  );

  RETURN QUERY
  SELECT 'alpha'::VARCHAR(50) AS variable, b[1][1] AS value
  UNION
  SELECT 'beta_purchases'::VARCHAR(50) AS variable, b[2][1] AS value
  UNION
  SELECT 'beta_age'::VARCHAR(50) AS variable, b[3][1] AS value;

  END;
  $$ LANGUAGE plpgsql;
\end{minted}

\section{Logistische Regression}
\label{appendix:E:3}

\begin{minted}[linenos,breaklines,frame=single]{sql}
  -- this procedure calculates the logits for current parameter values
  CREATE OR REPLACE FUNCTION calculate_logit()
  RETURNS void AS $$
  BEGIN

  DELETE FROM logits;

  WITH
    alpha_old AS (
      SELECT old
      FROM parameters
      WHERE variable = 'alpha'
    ),
    alpha_new AS (
      SELECT new
      FROM parameters
      WHERE variable = 'alpha'
    )
  INSERT INTO logits
    SELECT
      d.id,
      1 / (1 + EXP(- SUM(d.value * p.old) - (SELECT old FROM alpha_old))) AS old,
      1 / (1 + EXP(- SUM(d.value * p.new) - (SELECT new FROM alpha_new))) AS new
    FROM datapoints d
    JOIN parameters p ON p.variable = d.variable
    GROUP BY d.id;

  RETURN;

  END;
  $$ LANGUAGE plpgsql;

  -- this procedure calculates the gradient for the current parameter values
  CREATE OR REPLACE FUNCTION calculate_gradient()
  RETURNS void AS $$
  BEGIN

  DELETE FROM gradient;

  -- calculate gradient for alpha
  INSERT INTO gradient
  SELECT 'alpha' AS variable, SUM(bv.value - l.old) AS value
  FROM logits l
  JOIN binary_values bv ON bv.id = l.id;

  -- calculate other gradients
  INSERT INTO gradient
  SELECT d.variable, SUM(d.value * (bv.value - l.old)) AS value
  FROM logits l
  JOIN binary_values bv ON bv.id = l.id
  JOIN datapoints d ON d.id = l.id
  GROUP BY d.variable;

  RETURN;

  END;
  $$ LANGUAGE plpgsql;

  -- this procedure calculates the new parameters
  CREATE OR REPLACE FUNCTION calculate_new_parameters(step NUMERIC(65, 30))
  RETURNS void AS $$
  BEGIN

  UPDATE parameters
  SET new = old + step * gradient.value
  FROM gradient
  WHERE gradient.variable = parameters.variable;

  RETURN;

  END;
  $$ LANGUAGE plpgsql;

  -- this procedure calculates the log likelihood function for current parameter values
  -- and states if the new parameters are really better
  CREATE OR REPLACE FUNCTION are_new_parameters_better()
  RETURNS BOOLEAN AS $$
  DECLARE
    better BOOLEAN;
  BEGIN

  SELECT
    SUM(LOG(bv.value * l.new + (1 - bv.value) * (1 - l.new))) >
    SUM(LOG(bv.value * l.old + (1 - bv.value) * (1 - l.old))) INTO better
  FROM logits l
  JOIN binary_values bv ON bv.id = l.id;

  RETURN better;

  END;
  $$ LANGUAGE plpgsql;

  -- main procedure for execution
  CREATE OR REPLACE FUNCTION logistic_regression(number_datapoints INTEGER, rounds INTEGER, step NUMERIC(65, 30))
  RETURNS TABLE (
    variable VARCHAR(50),
    value NUMERIC(65, 30)
  ) AS $$
  DECLARE
    better BOOLEAN;
    counter INTEGER;
  BEGIN

  -- create a temporary table for the data
  DROP TABLE IF EXISTS datapoints;
  CREATE TEMPORARY TABLE datapoints (
    id INTEGER,
    variable VARCHAR(50),
    value NUMERIC(65, 30)
  );

  -- insert all linear transformed values for column 'money' into data
  INSERT INTO datapoints
  SELECT
    row_number() OVER () AS id,
    'beta_money' AS variable,
    (money - (
      SELECT MIN(money) FROM sample
    ))::NUMERIC / ((
      SELECT MAX(money) FROM sample
    ) - (
      SELECT MIN(money) FROM sample
    ))::NUMERIC AS value
  FROM sample
  LIMIT number_datapoints;

  -- create temporary table for binary variable
  DROP TABLE IF EXISTS binary_values;
  CREATE TEMPORARY TABLE binary_values (
    id INTEGER,
    value INTEGER
  );

  -- insert all values for column 'prime' into binary_values
  INSERT INTO binary_values
  SELECT
    row_number() OVER () AS id,
    prime AS value
  FROM sample
  LIMIT number_datapoints;

  -- create temporary table for parameters
  DROP TABLE IF EXISTS parameters;
  CREATE TEMPORARY TABLE parameters (
    variable VARCHAR(50),
    old NUMERIC(65, 30),
    new NUMERIC(65, 30)
  );

  -- set initial parameters
  INSERT INTO parameters VALUES
    ('alpha', 0, 0),
    ('beta_money', 0, 0);

  -- create temporary table for logits
  DROP TABLE IF EXISTS logits;
  CREATE TEMPORARY TABLE logits (
    id INTEGER,
    old NUMERIC(65, 30),
    new NUMERIC(65, 30)
  );

  -- insert initial values into logit table
  PERFORM calculate_logit();

  -- create temporary table for gradient
  DROP TABLE IF EXISTS gradient;
  CREATE TEMPORARY TABLE gradient (
    variable VARCHAR(50),
    value DECIMAL(65, 30)
  );

  -- insert variables in gradient table
  INSERT INTO gradient VALUES
    ('alpha', 0),
    ('beta_money', 0);

  -- loop
  counter := 0;
  WHILE counter < rounds AND step > 0.000000000000000000000000000001 LOOP

    PERFORM calculate_gradient();
    PERFORM calculate_new_parameters(step);
    PERFORM calculate_logit();
    better := are_new_parameters_better();

    WHILE NOT better AND step > 0.000000000000000000000000000001 LOOP

      step := step / 2;
      PERFORM calculate_new_parameters(step);
      PERFORM calculate_logit();
      better := are_new_parameters_better();

    END LOOP;

    UPDATE parameters
    SET old = new;

    UPDATE logits
    SET old = new;

    counter := counter + 1;

  END LOOP;

  UPDATE parameters
  SET old = old / ((SELECT MAX(money) FROM sample) - (SELECT MIN(money) FROM sample))
  WHERE parameters.variable = 'beta_money';

  UPDATE parameters
  SET old = old - (SELECT old FROM parameters p2 WHERE p2.variable = 'beta_money') * (SELECT MIN(money) FROM sample)
  WHERE parameters.variable = 'alpha';

  RETURN QUERY
  SELECT parameters.variable::VARCHAR(50), old AS value
  FROM parameters;

  END;
  $$ LANGUAGE plpgsql;
\end{minted}

\chapter{Python-Skripte für das Benchmarking}
\label{appendix:F}

\section{Berechnung der Benchmarks}
\label{appendix:F:1}

\begin{minted}[linenos,breaklines,frame=single]{python}
  from time import time
  from subprocess import call
  import os
  import sys
  import csv
  import json

  def print_progessbar(iterations, progress):
    sys.stdout.write("\033[100D")
    for i in range(progress * 50 // iterations): sys.stdout.write("#")
    for i in range(50 - (progress * 50 // iterations)): sys.stdout.write(".")
    sys.stdout.write("  ||  %i%% / 100%%" % (progress * 100 // iterations))
    sys.stdout.flush()

  def benchmark(type_regression, language, command, set_number_datapoints, file, iterations):
    print("Evaluating %s in %s..." % (type_regression, language))
    for number_datapoints in set_number_datapoints:
      print("Use %s datapoints:" % (number_datapoints))
      logfile = open("logs/%s-%s-%i.txt" % (language, type_regression.replace(" ", "-"), number_datapoints), "a")
      if command.count("%") == 2:
        exec_command = command % (number_datapoints, 8 / number_datapoints)
      else:
        exec_command = command % number_datapoints
      for i in range(iterations):
        print_progessbar(iterations, i)
        start_time = time()
        call(exec_command, stdout = logfile, stderr = logfile, shell = True)
        end_time = time()
        file.write("%s,%s,%i,%s\n" % (
          language,
          type_regression,
          number_datapoints,
          (end_time - start_time)
        ))
      logfile.close()
      print_progessbar(iterations, iterations)
      print("")
    return

  def main(argv):
    iterations = 100
    set_number_datapoints = [10, 100, 1000, 10000, 100000]
    run_r = False
    run_tensorflow = False
    run_mysql = False
    run_postgresql = False
    run_simple_linear_regression = False
    run_multiple_linear_regression = False
    run_logistic_regression = False

    for arg in argv:
      if "--datapoints=" in arg: set_number_datapoints = [int(arg.split("=")[1])]
      if "--iterations=" in arg: iterations = int(arg.split("=")[1])
      if arg in ["--r", "-r"]: run_r = True
      if arg in ["--tensorflow", "-t"]: run_tensorflow = True
      if arg in ["--mysql", "-m"]: run_mysql = True
      if arg in ["--postgresql", "-p"]: run_postgresql = True
      if arg in ["--simple-linear", "-slr"]: run_simple_linear_regression = True
      if arg in ["--multiple-linear", "-mlr"]: run_multiple_linear_regression = True
      if arg in ["--logistic", "-lr"]: run_logistic_regression = True

    if not ((
      run_r or
      run_tensorflow or
      run_mysql or
      run_postgresql
    ) and (
      run_simple_linear_regression or
      run_multiple_linear_regression or
      run_logistic_regression
    )):
      run_r = True
      run_tensorflow = True
      run_mysql = True
      run_postgresql = True
      run_simple_linear_regression = True
      run_multiple_linear_regression = True
      run_logistic_regression = True

    file = open("benchmarks-%i.csv" % (time()), "w")
    file.write("language,type,datapoints,time\n")

    if run_r and run_simple_linear_regression:
      benchmark(
        "simple linear regression",
        "r",
        "Rscript r/simpleLinearRegression.R %i -",
        set_number_datapoints,
        file,
        iterations
      )
    if run_r and run_multiple_linear_regression:
      benchmark(
        "multiple linear regression",
        "r",
        "Rscript r/multipleLinearRegression.R %i -",
        set_number_datapoints,
        file,
        iterations
      )
    if run_r and run_logistic_regression:
      benchmark(
        "logistic regression",
        "r",
        "Rscript r/logisticRegression.R %i -",
        set_number_datapoints,
        file,
        iterations
      )
    if run_tensorflow and run_simple_linear_regression:
      benchmark(
        "simple linear regression",
        "tensorflow",
        "python3 tensorflow/simpleLinearRegression.py %i -",
        set_number_datapoints,
        file,
        iterations
      )
    if run_tensorflow and run_multiple_linear_regression:
      benchmark(
        "multiple linear regression",
        "tensorflow",
        "python3 tensorflow/multipleLinearRegression.py %i -",
        set_number_datapoints,
        file,
        iterations
      )
    if run_tensorflow and run_logistic_regression:
      benchmark(
        "logistic regression",
        "tensorflow",
        "python3 tensorflow/logisticRegression.py %i -",
        set_number_datapoints,
        file,
        iterations
      )
    if run_mysql and run_simple_linear_regression:
      benchmark(
        "simple linear regression",
        "mysql",
        "echo \"CALL regression.simple_linear_regression(%i)\" | " + "mysql -u %s -p%s" % (
          json.loads(os.environ["MYSQL_CONFIG"])["user"],
          json.loads(os.environ["MYSQL_CONFIG"])["password"]
        ),
        set_number_datapoints,
        file,
        iterations
      )
    if run_mysql and run_multiple_linear_regression:
      benchmark(
        "multiple linear regression",
        "mysql",
        "echo \"CALL regression.multiple_linear_regression(%i)\" | " + "mysql -u %s -p%s" % (
          json.loads(os.environ["MYSQL_CONFIG"])["user"],
          json.loads(os.environ["MYSQL_CONFIG"])["password"]
        ),
        set_number_datapoints,
        file,
        iterations
      )
    if run_mysql and run_logistic_regression:
      benchmark(
        "logistic regression",
        "mysql",
        "echo \"CALL regression.logistic_regression(%i, 1000, %f)\" | " + "mysql -u %s -p%s" % (
          json.loads(os.environ["MYSQL_CONFIG"])["user"],
          json.loads(os.environ["MYSQL_CONFIG"])["password"]
        ),
        set_number_datapoints,
        file,
        iterations
      )
    if run_postgresql and run_simple_linear_regression:
      benchmark(
        "simple linear regression",
        "postgresql",
        "echo \"SELECT simple_linear_regression(%i)\" | psql regression",
        set_number_datapoints,
        file,
        iterations
      )
    if run_postgresql and run_multiple_linear_regression:
      benchmark(
        "multiple linear regression",
        "postgresql",
        "echo \"SELECT multiple_linear_regression(%i)\" | psql regression",
        set_number_datapoints,
        file,
        iterations
      )
    if run_postgresql and run_logistic_regression:
      benchmark(
        "logistic regression",
        "postgresql",
        "echo \"SELECT logistic_regression(%i, 1000, %f)\" | psql regression",
        set_number_datapoints,
        file,
        iterations
      )

    file.close()
    return

  if __name__ == "__main__":
    main(sys.argv)
\end{minted}
