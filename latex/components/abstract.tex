% Abstract for the TUM report document
% Included by MAIN.TEX


\clearemptydoublepage
\phantomsection
\addcontentsline{toc}{chapter}{Zusammenfassung}





\vspace*{1cm}
\begin{center}
{\Large \bf Zusammenfassung}
\end{center}
\vspace{1cm}

Das Ziel der vorliegenden Arbeit ist es, die Durchführung von statistischen Methoden in relationalen Datenbanksystemen zu demonstrieren. Das statistische Konzept, dass dazu verwendet wird, ist die lineare und die logistische Regressionsanalyse. Dazu werden zuerst kurz die mathematischen Grundlagen erklärt. Danach wird die Umsetzung der Regressionsanalyse mit verschiedenen Programmiersprachen demonstriert, insbesondere in zwei relationalen Datenbanken. Diese Implementierungen werden daraufhin miteinander verglichen. Dabei stellt man fest, dass die lineare Regressionsanalyse bei kleinen Datenmengen in Datenbanken sehr gut funktioniert. Logistische Regression mit Gradientenverfahren kann dagegen nicht mehr mit anderen Implementierungen mithalten. Zum Abschluss wird ein mögliches Erweiterungspotenzial für relationale Datenbanksysteme erläutert, insbesondere eine Erweiterung um Matrix-Operationen.

\vspace*{2cm}
\begin{center}
{\Large \bf Abstract}
\end{center}
\vspace{1cm}

The goal of this thesis is to demonstrate the implementation of statistical methods in relational database systems. The statistical concepts used are linear and logistic regression analysis. First, the mathematical basics are explained briefly. Then the implementation of the regression analysis will be demonstrated with different programming languages, especially in two relational databases. These implementations are then compared. It is found that the linear regression analysis works very well for small amounts of data in relational databases. On the other hand, logistic regression with gradient descent can no longer compete with other implementations. Finally, an expansion potential for relational database systems is explained, in particular an extension to matrix operations.
